% Chapter 1

\chapter*{Introduction} % Main chapter title
\addcontentsline{toc}{chapter}{Introduction}
\label{Introduction} % For referencing the chapter elsewhere, use \ref{Chapter1} 

%----------------------------------------------------------------------------------------

% Define some commands to keep the formatting separated from the content 
%----------------------------------------------------------------------------------------
The cryptography used by most of the cryptocurrencies is mainly based on the
private-public key pair. It is therefore fundamental the method used to generate
private keys, which must be efficient, secure and suitable for the situation.
\\ \\
This thesis claim to analyze in detail the principal techniques used for the derivation of the public-private keys pair in the Bitcoin framework.
\\ \\
The first chapter it will give an explanation on the basic concepts that will be used for this work.
There are two fundamental elements used: HASH functions and the Elliptic Curve. The first are functions that are supposed to be irreversible, because it doesn't exist an analytic expression for the inverse. The only way to compute the inverse of a hash function is by trying and this will take to much time, due to high computational costs.
The second, the Elliptic Curve, is a plane algebraic curve defined by an equation, over a specific field. A point on this curve is called Public key, instead the integer number, used to obtain the point, is called private key. In this chapter all the most important properties of this curve will be explained.
\\ \\
In the second chapter will be analyzed in detail the principal techniques used in order to generate private and public key pairs. In particular we will see four type of derivations. The first and naive method consists of randomly extracting a number and consider it as a private key, then generate the corresponding public key each time a new pair is request. The other three methods that will be analyzed are the so called: \textit{deterministic}. This is due to the fact that in order to generate a bunch of keys, it is necessary one single datum, called \textit{seed}. These three methods are in a increasing scale of difficulty and complexity and we will see their principal advantage and disadvantage. The last type of derivation is the most used and derives the keys in a hierarchical way, this method will be seen in the next chapter.
\\ \\
The third chapter will be focus on the analysis of the Hierarchical Deterministic Wallets, the most sophisticated type of derivation used up to now. It is defined by BIP32 [\cite{1}] and it is used by most of the Bitcoin Wallet. This derivation is deterministic, a seed is needed, and it is hierarchical. From the seed it is possible to derive a large number of keys and all of these keys can derive new keys in the same way and so on. This procedure can be iterated as long as desired, leaving the user a wide choice in the derivation of these numbers.
\\ \\
In the fourth chapter there will be the analysis of two possible ways to store the seed: the first is the one proposed by BIP39 [\cite{2}] and it is the most used in the Bitcoin framework; the second is the one used by Electrum [\cite{3}], one of the principal Bitcoin wallet. Both of them used a mnemonic phrase, a sentence composed of a certain number of words from which it is possible to derive the seed. Nevertheless they have some differences and we will analyze them. The principal difference stands in the fact that both the methods have a way to verify the correctness of the mnemonic phrase. With BIP32 it is only possible to check if the phrase is plausible, instead with Electrum it is possible to assign a version to the seed that will be generated by the mnemonic phrase, giving a purpose to the keys generated from it.
\\ \\
The fifth chapter will be focused on some possible application of the Hierarchical Deterministic Wallet proposed by BIP32. In particular we will see the standard way to write a \textit{path}, in order to easily understand how to generate a particular key from the seed.  We will also analyze one of the standard used by most of the Bitcoin wallet: BIP43 [\cite{4}]. The purpose of this BIP is to give a particular meaning to some branches of the tree. We will therefore describe two important applications: multi-coin wallet BIP44 [\cite{5}] and SegWit addresses BIP49 [\cite{6}].
\\ \\
In the appendix there will be a summary of the methods used for the representation of the private and public keys in the Bitcoin framework and the respective addresses.
\\ \\
Along with this writing, I attach the github link of the repository of the python code for the course of the professor F. Ametrano. In this repository I have replicated in python all the procedures and methods presented and described in this thesis, neglecting all those parts that are not inherent to it and writing the important ones in a synthetic and essential way.
\\ \\
\hypersetup{
	colorlinks=true,
	urlcolor=black
}
\href{https://github.com/fametrano/BitcoinBlockchainTechnology}{\texttt{https://github.com/fametrano/BitcoinBlockchainTechnology}}
