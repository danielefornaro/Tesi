% Chapter 1

\chapter{Hierarchical Deterministic Wallet} % Main chapter title

\label{hd wallet} % For referencing the chapter elsewhere, use \ref{Chapter1}

%----------------------------------------------------------------------------------------

% Define some commands to keep the formatting separated from the content 
%\newcommand{\keyword}[1]{\textbf{#1}}
%\newcommand{\tabhead}[1]{\textbf{#1}}
%\newcommand{\code}[1]{\texttt{#1}}
%\newcommand{\file}[1]{\texttt{\bfseries#1}}
%\newcommand{\option}[1]{\texttt{\itshape#1}}

%----------------------------------------------------------------------------------------

In this chapter we will see how an HD wallet works.
\\ \\
\section{Elements}
Let's focus on the main elements of the Wallet:
\begin{itemize}[label=$\diamond$]
	\item Seed
	\item Extended keys
\end{itemize}

\subsection{Seed}
The entire Wallet is based on a \textit{seed}.
\\ \\
It is a number taken from a \textit{Discrete Uniform Random Variable}
\begin{equation*}
seed=X(\omega) \qquad X\sim \mathcal{U}(S)
\end{equation*}
Where $S$ is the finite set of natural number in the range from $1$ to an arbitrary value.\\ Obviously the greater the set from which the number can be extracted, the better it is for the security of the seed itself.
\\ \\
This is an example of seed expressed in hexadecimal format: \\
\textit{seed}=fffcf9f6f3f0edeae7e4e1dedbd8d5d2cfccc9c6c3c0bdbab7b4b1aeaba8a5a29f9c999 \\ 693908d8a8784817e7b7875726f6c696663605d5a5754514e4b484542 

\subsection{Extended Key}
An Extended Key is a sequence of bytes, encoded in base58. It contains all the information necessary for the derivation. \\ \\
Once it is decoded we will obtain exactly 78 bytes, with a specific meaning and order:
\begin{itemize}[label=$\circledast$]
	\item 4 bytes are used to specified the \textbf{version}.
	\item 1 byte is used to specified the \textbf{depth} in the hierarchical tree: the extended key derived directly from the seed has $depth=0$, its first children have $depth=1$, grandchildren have $depth=2$ and so on.
\end{itemize}

 
\section{Functional explanation}

\section{Code implementation}
