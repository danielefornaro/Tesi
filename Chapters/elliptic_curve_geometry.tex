% Chapter 1

\chapter{Elliptic Curve Geometry} % Main chapter title

\label{EC} % For referencing the chapter elsewhere, use \ref{Chapter1} 

%----------------------------------------------------------------------------------------

% Define some commands to keep the formatting separated from the content 
\newcommand{\keyword}[1]{\textbf{#1}}
\newcommand{\tabhead}[1]{\textbf{#1}}
\newcommand{\code}[1]{\texttt{#1}}
\newcommand{\file}[1]{\texttt{\bfseries#1}}
\newcommand{\option}[1]{\texttt{\itshape#1}}

%----------------------------------------------------------------------------------------

\section{Introduction}
Bitcoin security is based on public and private key cryptograpy. The main concept is that it is simple to compute the public key, knowing the private, but it is infeasible to calculate the private key, knowing the public. \\ \\
In order to obtain this result a particular Elliptic Curve is used.

\section{Point on an Elliptic Curve}

A point $Q$, which coordinates are $x$ and $y$, belong to an Elliptic Curve if and only if $Q$ satisfies the following equation:
\begin{equation}\label{GeneralEC}
y^2=x^3+ax+b
\end{equation}
over a certain field $\mathbb{F}_p$, where $a$ and $b$ are the coefficients of the curve. Here we want to analyse only the  \\ \\
The curve is specified by the definition of the coefficients and the field $\mathbb{F}_p$ should be consider for simplicity the 

\subsection{Bitcoin Elliptic Curve}
Bitcoin uses a specific Elliptic Curve defined over the finite field of the natural numbers, where $a=0$ and $b=7$. \\ \\
The equation \ref{GeneralEC} becomes:

\begin{equation}\label{BitcoinEC}
y^2=x^3+7 \quad \textrm{mod} \ p
\end{equation}

The \textit{mod p} (modulo prime number) indicates that this curve is over a finite field of prime order $p$, where $p=2^{256}-2^{32}-2^9-2^8-2^7-2^6-2^4-1$

\subsection{Proprierties}
A point on the Bitcoin Elliptic Curve has some particular proprierties:
\begin{itemize}
	\item Symmetry
	\item Point addition
	\item Scalar multiplication
\end{itemize}

\subsubsection{Symmetry}
For every point in the $x$ axis exists two points in the $y$ axis. Suppose that a point $P(x,y)$ belongs to the Elliptic Curve, then it must satisfy the equation \ref{BitcoinEC}.
So it is easy to prove that the point $Q(x,p-y)$ belongs to the curve too.

\subsubsection{Point addition}
Suppose that \textit{A}, \textit{B} and \textit{C} are points of an Elliptic Curve that satisfy

\begin{equation}\label{Point addition}
A+B+C=0 \quad \textrm{mod} \ p
\end{equation}








