% Chapter 1

\chapter{Mnemonic phrase} % Main chapter title

\label{mnemonic} % For referencing the chapter elsewhere, use \ref{Chapter1}

%----------------------------------------------------------------------------------------

% Define some commands to keep the formatting separated from the content 
%\newcommand{\keyword}[1]{\textbf{#1}}
%\newcommand{\tabhead}[1]{\textbf{#1}}
%\newcommand{\code}[1]{\texttt{#1}}
%\newcommand{\file}[1]{\texttt{\bfseries#1}}
%\newcommand{\option}[1]{\texttt{\itshape#1}}

%----------------------------------------------------------------------------------------

We have seen how it is possible to generate keys starting from a seed. But a seed is a long number, difficult to remember and not easy to write down on a paper. You may incur typos while transcribing it, and this can compromise the entire wallet.
\begin{remark}
	Mistyping a single digit in the seed produce completely different keys
\end{remark}
In order to work around this problem, some solution were implemented. Among them, the most widespread and used is the one described by BIP39, \textit{Bitcoin Improvement Proposal}. This is not the only one, in this chapter we will also see another solution proposed by Electrum, one of the most famous Bitcoin wallet.
\\ \\
Both these solutions used a \textit{Mnemonic phrase}, from which the seed is obtained. This sentence is designed to avoid typing errors while maintaining the same level of security and entropy.
\begin{flushleft}
	\textbf{What is a Mnemonic phrase?}
\end{flushleft}
A Mnemonic phrase is a set of words taken from a specific dictionary. Although the choice of the dictionary is not binding, the most commonly used among the practitioners is the English one, defined by BIP39. It contains 2048 common words of the English language, each of them from 3 to 9 letters. The set of words that make up the dictionary must be chosen in such a way that the words within it are easy to remember and difficult to misinterpret with one another. It is better to avoid to insert into the dictionary two words with similar meaning or spelling.

\section{BIP39}
First we will see how to generate a mnemonic phrase in the framework of BIP39 and then how it is possible to obtain a seed from it.

\subsection{Mnemonic Generation}
In order to generate a Mnemonic phrase we will start from a given entropy, that can be seen as a large integer number. The way to obtain it can be left free to the user: he can obtain it by inserting arbitrarily chosen numbers (poor choice of randomness), roll a dice many times or with any other method he considers suitable. Many softwares provide a function that generate entropy with quite randomness, but if someone is skeptical about the reliability of software randomness, he must provide himself with such integer number.
\\ \\
Call \textit{ENT} the number of binary digits of the given entropy. Then \textit{ENT} should belongs to a given set:
\begin{equation*}
	ENT \in \{128,160,192,224,256\}
\end{equation*}
The reason for a given length for the entropy will be clear in a moment.
\\ \\
Write the entropy in bytes format, obtaining a string of $ENT/8$ length. Compute the SHA256 algorithm and consider only the first $ENT/32$ bits as a checksum. Finally add these bits to the bottom of the entropy, obtaining an integer number, called \textit{entropy\_checked}, expressed in binary format of length equal to: $ENT+ENT/32$.

\begin{flushleft}
	In python:
\end{flushleft}

\begin{lstlisting}[language=Python]
from hashlib import sha256

entropy_bytes = entropy.to_bytes(int(ENT/8), byteorder='big')
checksum = sha256(entropy_bytes).digest()
entropy_checked = entropy_bin + checksum_bin[:int(ENT/32)]
\end{lstlisting}
Where \textit{entropy\_bin} and \textit{checksum\_bin} are strings of bits that can be concatenated.
\\ \\
Now it is clear the reason for a constraint on the length of the entropy in input: 
\begin{enumerate}[label=\roman*]
	\item \textit{ENT} must be a dividend of 32
	\item $ENT<128$ could be not secure enough.
	\item $ENT>256$ is useless.
\end{enumerate}
The point (i) is due to the structure proposed by BIP39. It is only a convention to take the first $ENT/32$ bits as a checksum. However it is essential that the final length of the entropy plus the checksum must be a dividend of 11, from the moment that the dictionary is a set of $2^{11}$ words.
\\ \\
The point (ii) is just a suggestion because taking less entropy could bring to a leak of security. It will be easier for an attacker to guess your mnemonic phrase by trying out all the possible combinations if less words are involved. It is important to remember that adding even a single bit of entropy, doubles the difficulty of guessing it.
\\ \\
The point (iii) is another suggestion. A private key is a number smaller then $2^{256}$ therefore it would be useless to generate a seed starting from an entropy with more then 256 bits.
\\ \\
Thanks to constraint (i) we obtain that the length of \textit{entropy\_checked} is a dividend of 11:
\begin{equation*}
len(entropy\_checked)=ENT+\dfrac{ENT}{32}=\dfrac{33}{32}\cdot ENT=11\cdot \dfrac{3}{32}\cdot ENT
\end{equation*}
Consider \textit{entropy\_checked} as a string of bits and divide it in substring, each of 11 bits length, obtaining $(\frac{3}{32}\cdot ENT)$ strings of bits.
\\ \\
Each of these strings represents an integer number that can take values in the range between 0 and 2047, ie $2^{11}-1$. Associate each of these numbers with a word in the chosen dictionary, suppose to consider the English one sorted alphabetically. Write down all these words, separated by a space and obtain the Mnemonic Phrase.
\\ \\
An example of Mnemonic Phrase (24 words):
\begin{center}
	\textit{inherit antique flame enrich tell arena eternal floor equal invite swarm pioneer oak benefit giggle damage ship shoe kitten zone shock decline kiss subject}
\end{center}


\subsection{From Mnemonic to Seed}
Once obtaining the Mnemonic phrase we need to derive the seed. In order to do so an hash function is used. So it will be infeasible to derive the Mnemonic phrase from the seed.
\\ \\
The function used is the PBKDF2 and it is used in order to avoid brute force attack, from the moment that the output has exactly the same length of a standard hash function, but it will take more times to calculate it from the moment that it will compute the same hash function many times.
\\ \\

It receives as input:
\begin{itemize}[label=$\odot$]
	\item \textbf{Password}: Mnemonic phrase
	\item \textbf{Salt}: 'mnemonic' + passphrase
	\item \textbf{Number of iterations}: 2048
	\item \textbf{Digest-module}: SHA512
	\item \textbf{Mac-module}: HMAC
\end{itemize}
Summing up it can be said that it calculates the same hash function (HMAC-SHA512) 2048 times.
\\ \\
In order to introduce more complexity in the seed computation a \textit{Salt} is introduced. If not specified the standard salt is simply the world 'mnemonic', otherwise it could be extended with an optional \textit{passphrase}.
\\ \\
Although it is true that a human being is a scarce source of randomness, the passphrase is usually chosen by the user. This is due to the fact that it has a different purpose to the mnemonic phrase. We will analyze some useful applications in the next chapter. 
\begin{remark}
	The randomness should be guaranteed by the input entropy used to generate the mnemonic phrase, not by the passphrase.
\end{remark}
The python code is the follow:
\begin{lstlisting}[language=Python]
from hashlib import sha512
from pbkdf2 import PBKDF2
import hmac

seed = PBKDF2(mnemonic, 'mnemonic' + passphrase, iterations = 2048, macmodule = hmac, digestmodule = sha512).read(64)
\end{lstlisting}
Where \textit{mnemonic} is the mnemonic phrase previously computed and \textit{passphrase} is chosen by the user

\begin{remark}
	With this procedure we always produce a seed of specific length: 512 bits. It will always be enough because every private key can take value from a smaller set of value (1 to \textit{order}).
\end{remark}

\section{Electrum Mnemonic}
Even if BIP39 is proposed, it is not the only solution adopted among the practitioners. One example is the one proposed by Electrum.
\\ \\
The main difference is in the way that the mnemonic phrase is generated and the purpose of it. Electrum choses to assign a version to the seed in such a way that is possible to recognize the purpose of the key generated from it.

\subsection{Mnemonic and Seed Generation}
Whenever a new mnemonic phrase is required, Electrum starts from an given entropy, generated through a random function. Obviously it is possible to generate a valid mnemonic phrase with an entropy chosen by the user, if he is skeptical or doesn't want to relay on the reliability of the randomness of a random function.
\\ \\
To be consistent, call \textit{ENT} the number of binary digits of the given entropy. Then \textit{ENT} must be a multiple of 11. 





