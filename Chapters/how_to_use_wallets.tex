% Chapter 1

\chapter{How to use a HD Wallet} % Main chapter title

\label{howto} % For referencing the chapter elsewhere, use \ref{Chapter1}

%----------------------------------------------------------------------------------------

% Define some commands to keep the formatting separated from the content 
%\newcommand{\keyword}[1]{\textbf{#1}}
%\newcommand{\tabhead}[1]{\textbf{#1}}
%\newcommand{\code}[1]{\texttt{#1}}
%\newcommand{\file}[1]{\texttt{\bfseries#1}}
%\newcommand{\option}[1]{\texttt{\itshape#1}}

%----------------------------------------------------------------------------------------
As already mentioned there are various way to use a Hierarchical Deterministic Wallet. It is possible to use this sequence of keys also for a non-monetary purpose, outside the cryptocurrency world. In fact these keys are simple number and they can be used for every possible purpose. Here we will focus only on the cryptocurrency application, in particular Bitcoin.

\section{Derivation path}
In order to easily recognize the path of a particular derivation, a common notion was introduced for the Hierarchical Deterministic Wallet.
\\ \\
Let's denote the extended master private key with \textit{m} and the extended master public key with \textit{M}.
\\ \\
The first normal private child derived from \textit{m} will be denoted by the number $0$:
\begin{equation*}
m /0
\end{equation*}
The fourth normal private child of the first normal private child of the master private key will use the following notation:
\begin{equation*}
m /0/3
\end{equation*}
In order to indicate a hardened child, the number of the index is followed by an apostrophe. The first hardened child will have $index=2^{31}$, the second will have $index=2^{31}+1$ and so on, but for the nomenclature of the path, $2^{31}$ will be omitted and the first hardened child of the master key will be written in this way: 
\begin{equation*}
m /0' 
\end{equation*}
It is possible to mix hardened and normal derivation. For example the $15^{th}$ hardened child of the $37^{th}$ normal child of the $6^{th}$ normal child of the $1019^{th}$ hardened child of \textit{m} will be represented in the following way:
\begin{equation*}
m /1018'/5/36/14'
\end{equation*}

\begin{remark}
	Although it is not recommended to use hardened derivation after a normal derivation, it is always possible to do so.
\end{remark}
This nomenclature can be used also to represent extended public keys. In fact it is possible to derive, with the normal derivation, an extended public key from the parent extended public key. The notation will be the usual one: for example the third public key derived from the sixth public key derived from \textit{M} will be represented in this way:
\begin{equation*}
M /2/5
\end{equation*}

\begin{remark}
It is useless to specify that children derived from M are normal and not hardened. It is impossible to use the hardened derivation to derive the public key from a parent public key.
\end{remark}

  
\section{BIP 43}
This BIP introduces a "Purpose Field" in the Hierarchical Deterministic wallet [\cite{4}]. The first child of the extended master private key specify the purpose of the entire branch.
\begin{equation*}
m\; \;/\;\; purpose'\;\; /\;\; *
\end{equation*}
For example if $purpose=44$ it means that this is a multi-coin wallet, if $purpose=49$ it means that the keys generated follow the BIP49 specification (P2WPKH nested in P2SH).

\subsection{Multi-coin wallet BIP 44}
The derivation scheme for a multi-coin wallet [\cite{5}] should be the follow:
\begin{equation*}
m / purpose' / coin\_type' / account' / change / address\_index
\end{equation*}
Let's see the meaning of each field:
\begin{itemize}
	\item \textit{Purpose}: it must be equal to 44' (or 0x8000002C)  and it indicates that the subtree of this node is used according to this specification. Hardened derivation is used at this level.
	\item \textit{Coin\_type}: this level creates a separate subtree for every cryptocoin, avoiding reusing addresses across cryptocoins and improving privacy issues. Coin type is a constant, set for each cryptocoin. Hardened derivation is used at this level.
	\\ \\
	Some example:
	\begin{center}
		\begin{tabular}{|| c | c ||} 
			\hline
			Path & Cryptocoin  \\ [0.5ex] 
			\hline\hline
			m / 44' / 0' & Bitcoin (mainnet) \\ 
			m / 44' / 1' & Bitcoin (testnet) \\			
			m / 44' / 2' & Litecoin \\			
			m / 44' / 3' & Dogecoin \\			
			m / 44' / 60' & Ethereum  \\ 
			m / 44' / 128' & Monero  \\
			m / 44' / 144' & Ripple  \\ 
			m / 44' / 1815' & Cardano  \\  
			$\vdots $& $\vdots $  \\ 
			\hline
		\end{tabular}
	\end{center}
	A list with the complete set of cryptocoins is available and in continuous update: \\ \url{https://github.com/satoshilabs/slips/blob/master/slip-0044.md }
	
	\item \textit{Account}: this level splits the key space into independent user identities, so the wallet never mixes the coins across different accounts.
	Users can use these accounts to organize the funds in the same fashion as bank accounts; for donation purposes, for saving purposes, for common expenses etc.
	Hardened derivation is used at this level.
	\item \textit{Change}: it can take only two value: 0 and 1. 0 is used for addresses that are meant to be visible outside of the wallet (e.g. for receiving payments). 1 is used for addresses which are not meant to be visible outside of the wallet and is used for return transaction change.
	Normal derivation is used at this level.
	\item \textit{Index}: this is the last derivation, used to have many keys for each cryptocoin. It can take values from 0 in sequentially increasing manner. Normal derivation is used at this level.
\end{itemize}
The principal advantage of this BIP is the possibility to easily back up all the cryptocoins of the user just by remembering a particular set of world (BIP39 mnemonic phrase). If these method is used to store more then one coins, keep attention in not to lose the master private key, otherwise all coins were lost.

\subsection{SegWit addresses BIP 49}
From August 2017 a new way to sign a Bitcoin transaction has been allowed, Segregated Witness: SegWit. It is not the purpose of this thesis to look inside of this topic, but it is important to say that a software not updated is not able to spend coins received in this format. So it is necessary to have a different branch in the derivation scheme just for SegWit addresses, in such a way that only software aware of SegWit are able to spend thus coins.
\\ \\
The derivation scheme introduced by BIP49 [\cite{6}] is the follow:
\begin{equation*}
m / 49' / coin\_type' / account' / change / address\_index
\end{equation*}
The logic of BIP44 was followed and it let the user the possibility to have a multi-coin wallet just by choosing a specific \textit{coin\_type}.
\\ \\
\begin{remark}
	In order to make SegWit a soft fork (backward compatible), a SegWit transaction is nested in a pay-to-script-hash, so the corresponding address must begin with '3'.	
\end{remark}




