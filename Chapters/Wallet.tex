% Chapter 1

\chapter{Wallet} % Main chapter title

\label{bip32} % For referencing the chapter elsewhere, use \ref{Chapter1}

%----------------------------------------------------------------------------------------

% Define some commands to keep the formatting separated from the content 
%\newcommand{\keyword}[1]{\textbf{#1}}
%\newcommand{\tabhead}[1]{\textbf{#1}}
%\newcommand{\code}[1]{\texttt{#1}}
%\newcommand{\file}[1]{\texttt{\bfseries#1}}
%\newcommand{\option}[1]{\texttt{\itshape#1}}

%----------------------------------------------------------------------------------------


A Bitcoin wallet is a structure used to store keys. \\ \\
There are different type of wallet:
\begin{itemize}
	\item Nondeterministic (\textit{random}) Wallet
	\item Deterministic Wallet
\end{itemize}

\begin{remark}
	Bitcoin wallets contains keys, not coins. Coins are in the Blockchain.
\end{remark}

\section{Nondeterministic (\textit{random}) Wallet}
A nondeterministic wallet is the simplest type of wallet. Each Key is randomly and independently generated.

\begin{enumerate}[label=(\roman*)]
	\item Consider a \textit{Discrete Uniform Random Variable}
	\begin{equation*}
		X\sim \mathcal{U}(S)
	\end{equation*}
	Where $S$ is the finite set of natural number in the range from $1$ to $order$.
	\item Take some realizations $k_1,k_2...k_n$ of $X$ using enough entropy to make these numbesr (\textit{private keys}) impossible to guess.
	\begin{equation*}
		k_1=X(\omega_1) \quad  k_2=X(\omega_2) \quad ... \quad k_n=X(\omega_n)
	\end{equation*}
	\item Go back to point (i) every time new \textit{private keys} are needed.
\end{enumerate}
With this procedure it is impossible to compute the \textit{public key} without having already the \textit{private key}.

\subsubsection{Pros and Cons}
Let's focus on the good and bad aspects of this wallet.

\begin{center}
	\begin{tabular}{ |p{6cm}|p{6cm}|  }
	\hline
	\multicolumn{2}{|c|}{\textbf{\textit{Random Wallet}}} \\
	\hline \hline 
	\\
	\centerline{\textbf{Pros}}&\centerline{\textbf{Cons}}\\
	\hline
	\begin{itemize}
		\item Easy to implement
	\end{itemize} &
	\begin{itemize}
		\item Difficult to find \underline{real} new entropy for every new \textit{private key}.
		\item Every time new \textit{private keys} are needed, you need to make new back up.
		\item Difficult to store or back up in a \textit{non digital way}. Awkward to write it down all yours keys on a paper.
	\end{itemize}\\
	\hline
\end{tabular}
\end{center}

The use of \textit{random wallet} is strongly discouraged for anything other than simple test. There are no good reason to use it.

\section{Deterministic Wallets}
A deterministic wallet is a more sophisticated one, in which every key is generated from a common "\textit{seed}". This means that knowing the \textit{seed} means also to know all the keys in the wallet.\\ \\
There are different types of deterministic wallets, in this text we will analyze three main types:
\begin{itemize}
	\item Deterministic Wallet \textit{type 1}
	\item Deterministic Wallet \textit{type 2}
	\item Hierarchical Deterministic Wallet
\end{itemize}
These wallet are in increasing order of complexity.

\subsection{Deterministic Wallet \textit{type-1}}
The Deterministic Wallet \textit{type-1} is one of the simplest Wallet among the deterministic ones. Each key is generated adding a number in a sequential order to the \textit{seed} and then computing an \textit{hash} function such as the \textbf{SHA256} function.
\\ \\
Let's see how to generate the $n^{th}$ private key:

\begin{enumerate}[label=(\roman*)]
	\item Generate a \textit{seed} (only once), a random number from a \textit{Discrete Uniform Random Variable}
	\begin{equation*}
	seed=X(\omega) \qquad X\sim \mathcal{U}(S)
	\end{equation*}
	Where $S$ is the finite set of natural number in the range from $1$ to $order$.
	\item Consider the numbers $seed$ and $n$ as strings and concatenate $n$ to $seed$, obtaining a $value$
	\begin{equation*}
	value=seed|n
	\end{equation*}
	\item Compute the SHA256 function to $value$ and obtain the $n^{th}$ \textit{private key}.
	\item Go back to point (ii) every time new \textit{private keys} are needed with $n=n+1$. 
\end{enumerate}
With this procedure it is impossible to compute the \textit{public key} without having already computed the \textit{private key}.

\subsubsection{Pros and Cons}
Let's focus on the good and bad aspects of this wallet.

\begin{center}
	\begin{tabular}{ |p{6cm}|p{6cm}|  }
		\hline
		\multicolumn{2}{|c|}{\textbf{\textit{Deterministic Wallet type-1}}} \\
		\hline \hline 
		\\
		\centerline{\textbf{Pros}}&\centerline{\textbf{Cons}}\\
		\hline
		\begin{itemize}
			\item In order to make a back up of the entire wallet it is needed to store the \textit{seed} only. All \textit{private keys} can be derived from it.
			\item A single back up is needed.
			\item The \textit{seed} can be stored also in a \textit{non digital way}, in a paper for example.
		\end{itemize} &
		\begin{itemize}
			\item Every time new \textit{public keys} are needed, you need to use the \textit{seed}, to compute new \textit{private keys} and then derive the \textit{public} ones. This could compromise the entire wallet, if the \textit{seed} is used in a non safe environment.
			\item There is only a \textit{key} sequence. No way to distinguish the "purpose" of each \textit{private key}.
		\end{itemize}\\
		\hline
	\end{tabular}
\end{center}
The use of this type of wallet is not recommended for everyday use, but it could be used to store Bitcoin in a safe place: \textit{cold wallet}. Without 


\section{Hierarchical deterministic Wallets}

\subsection{Key Concept}

\subsubsection{Seed}

\subsubsection{Extended Key}

\subsubsection{Mnemonic and Passphrase}

\subsection{Pros and Cons}

