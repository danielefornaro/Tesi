% Chapter 1

\chapter{Wallet} % Main chapter title

\label{wallet} % For referencing the chapter elsewhere, use \ref{Chapter1}

%----------------------------------------------------------------------------------------

% Define some commands to keep the formatting separated from the content 
%\newcommand{\keyword}[1]{\textbf{#1}}
%\newcommand{\tabhead}[1]{\textbf{#1}}
%\newcommand{\code}[1]{\texttt{#1}}
%\newcommand{\file}[1]{\texttt{\bfseries#1}}
%\newcommand{\option}[1]{\texttt{\itshape#1}}

%----------------------------------------------------------------------------------------

The way used to store keys is essential in private-public keys cryptography. For this reason different type of wallet were designed.
\begin{definition}
	A wallet is a software used to store keys. It is able also to sign messages with the private key, but in this framework we will only consider wallets as key containers. 
\end{definition}
There are different type of wallet:
\begin{itemize}
	\item Nondeterministic (\textit{random}) Wallet
	\item Deterministic Wallet
\end{itemize}

\begin{remark}
	Bitcoin wallets contains keys, not coins. Coins are in the Blockchain.
\end{remark}

\section{Nondeterministic (\textit{random}) Wallet}
A nondeterministic wallet is the simplest type of wallet. Each key is randomly and independently generated.

\begin{enumerate}[label=(\roman*)]
	\item Consider a \textit{Discrete Uniform Random Variable}
	\begin{equation*}
	X\sim \mathcal{U}(S)
	\end{equation*}
	Where $S$ is the finite set of natural number in the range from $1$ to $order$.
	\item Take some realizations $k_1,k_2...k_n$ of $X$ using enough entropy to make these numbers (\textit{private keys}) impossible to guess.
	\begin{equation*}
	k_1=X(\omega_1) \quad  k_2=X(\omega_2) \quad ... \quad k_n=X(\omega_n)
	\end{equation*}
	\item Go back to point (i) every time new \textit{private keys} are needed.
\end{enumerate}
With this procedure it is impossible to compute the \textit{public key} without having already computed the \textit{private key}.

\subsubsection{Pros and Cons}
Let's focus on the good and bad aspects of this wallet.

\begin{center}
	\begin{tabular}{ |p{6cm}|p{6cm}|  }
		\hline
		\multicolumn{2}{|c|}{\textbf{\textit{Random Wallet}}} \\
		\hline \hline 
		\begin{center}
			\textbf{Pros}
		\end{center}&\begin{center}
			\textbf{Cons}
		\end{center}\\
		\hline
		\begin{itemize}
			\item Easy to implement
		\end{itemize} &
		\begin{itemize}
			\item Difficult to find \underline{real} new entropy for every new \textit{private key}.
			\item Every time new \textit{private keys} are needed, a new back up is needed.
			\item Difficult to store or back up in a \textit{non digital way}. Awkward to write it down all yours keys on a paper.
		\end{itemize}\\
		\hline
	\end{tabular}
\end{center}
The use of \textit{random wallet} is strongly discouraged for anything other than simple test. There are no good reason to use it.

\section{Deterministic Wallets}
A deterministic wallet is a more sophisticated one, in which every key is generated from a common "\textit{seed}". This means that knowing the \textit{seed} leads to know also all the keys in the wallet.\\ \\
There are different types of deterministic wallets, in this text we will analyze three main types:
\begin{itemize}
	\item Deterministic Wallet \textit{type 1}
	\item Deterministic Wallet \textit{type 2}
	\item Deterministic Wallet \textit{type 3}
\end{itemize}
These wallet are in increasing order of complexity.

\subsection{Deterministic Wallet \textit{type-1}}
The Deterministic Wallet \textit{type-1} is one of the simplest wallet among the deterministic ones. Each key is generated adding a number in a sequential order to the \textit{seed} and then computing an \textit{hash} function, the \textbf{SHA256} function.
\\ \\
Let's see how to generate the $n^{th}$ private key:

\begin{enumerate}[label=(\roman*)]
	\item Generate a \textit{seed} (only once), a random number from a \textit{Discrete Uniform Random Variable}
	\begin{equation*}
	seed=X(\omega) \qquad X\sim \mathcal{U}(S)
	\end{equation*}
	Where $S$ is the finite set of natural number in the range from $1$ to $order$.
	\item Consider the numbers $seed$ and $n$ as strings and concatenate $n$ to $seed$, obtaining a $value$
	\begin{equation*}
	value=seed|n
	\end{equation*}
	\item Compute the SHA256 function to $value$ and obtain the $n^{th}$ \textit{private key}.
	\item Go back to point (ii) every time new \textit{private keys} are needed with $n=n+1$. 
\end{enumerate}
With this procedure it is impossible to compute the \textit{public key} without having already computed the \textit{private key}.

\subsubsection{Pros and Cons}
Let's focus on the good and bad aspects of this wallet.

\begin{center}
	\begin{tabular}{ |p{6cm}|p{6cm}|  }
		\hline
		\multicolumn{2}{|c|}{\textbf{\textit{Deterministic Wallet type-1}}} \\
		\hline \hline 
		\begin{center}
			\textbf{Pros}
		\end{center}&\begin{center}
			\textbf{Cons}
		\end{center}\\
		\hline
		\begin{itemize}
			\item In order to make a back up of the entire wallet it is needed to store the \textit{seed} only. All \textit{private keys} can be derived from it.
			\item A single back up is needed.
			\item The \textit{seed} can be stored easily also in a \textit{non digital way}, in a paper for example.
		\end{itemize} &
		\begin{itemize}
			\item Every time new \textit{public keys} are needed, you need to use the \textit{seed}, to compute new \textit{private keys} and then derive the \textit{public} ones. This could compromise the entire wallet, if the \textit{seed} is used in a non safe environment.
			\item There is only a \textit{key} sequence. No way to distinguish the "purpose" of each \textit{private key}.
		\end{itemize}\\
		\hline
	\end{tabular}
\end{center}
The use of this type of wallet is not recommended for everyday use, but it could be used to store Bitcoin in a safe place: \textit{cold wallet}. 

\subsection{Deterministic Wallet \textit{type-2}}
The Deterministic Wallet \textit{type-2} is more sophisticated. Each \textit{private key} is generated in such a way that it is possible to compute the respective \textit{public key} without knowing the \textit{private}.
\\ \\
First let's introduce the necessary ingredients:
\begin{itemize}[label=$\diamond$]
	\item \textbf{Master private key} ($mp$): a random number, generated from a \textit{Discrete Uniform Random Variable}
	\begin{equation*}
	mp=X(\omega) \qquad X\sim \mathcal{U}(S)
	\end{equation*}
	Where $S$ is the finite set of natural number in the range from $1$ to $order$. \\ The master private key must be take \underline{secret}.
	\item \textbf{Master public key} ($MP$): a point on the EC, obtained from the $mp$:
	\begin{equation*}
	MP=mp\cdot G
	\end{equation*}
	Where $G$ is the \textit{generator}.\\ This point can be consider \underline{non-secret}.
	\item \textbf{Public random number} ($r$): a random number, generated from a \textit{Discrete Uniform Random Variable}
	\begin{equation*}
	r=X(\omega) \qquad X\sim \mathcal{U}(S)
	\end{equation*}
	This number can be consider \underline{non-secret}.
\end{itemize}
Let's see how to generate the $n^{th}$ private key: $p_n$

\begin{enumerate}[label=(\roman*)]
	\item Compute the SHA256 function to the concatenation of $r$ with $n$, considered as string:
	\begin{equation*}
	h_{n|r}=SHA256 (n|r)
	\end{equation*}
	$h_{n|r}$ can be consider \underline{non-secret}, from the moment that it is derived from non secret information.
	\item Compute the $n^{th}$ \textit{private key} adding $mp$ to $h_{n|r}$:
	\begin{equation*}
	p_n=mp+h_{n|r} \qquad \mod (order)
	\end{equation*}
\end{enumerate}
In order to obtain the corresponding \textit{public key} $P_n$, it is possible to compute the standard multiplication:
\begin{equation*}
P_n=p_n \cdot G
\end{equation*}
It is also possible to compute $P_n$ without knowing $p_n$, using only non-secret information: $h_{n|r}$ and $MP$.
\begin{enumerate}[label=(\roman*)]
	\item Compute $V$:
	\begin{equation*}
	V=h_{n|r} \cdot G
	\end{equation*}
	$V$ can be see as the \textit{public key} of $h_{n|r}$ and can be consider \underline{non-secret}.
	\item Add $MP$ to $V$:
	\begin{equation*}
	P_n=MP+V
	\end{equation*}
	Where the sum in this contest is the one defined between two point in the EC.
\end{enumerate}
It is easy to prove that $P_n$ can be computed in these two way:
\begin{equation*} \label{eq1}
\begin{split}
P_n & = p_n \cdot G \\
& = (mp+h_{n|r}) \cdot G \\
& = (mp \cdot G) + (h_{n|r}\cdot G) \\
& = MP + V
\end{split}
\end{equation*}

\subsubsection{Pros and Cons}
Let's focus on the good and bad aspects of this wallet.

\begin{center}
	\begin{tabular}{ |p{6cm}|p{6cm}|  }
		\hline
		\multicolumn{2}{|c|}{\textbf{\textit{Deterministic Wallet type-2}}} \\
		\hline \hline 
		\begin{center}
			\textbf{Pros}
		\end{center}&\begin{center}
			\textbf{Cons}
		\end{center}\\
		\hline
		\begin{itemize}
			\item In order to make a back up of the entire wallet it is needed to store the \textit{master private key} and the random number $r$. All \textit{private keys} can be derived from them.
			\item A single back up is needed.
			\item The \textit{master private key} can be stored easily also in a \textit{non digital way}, in a paper for example.
			\item It is possible to derive a new \textit{public key} using only non-secret information, with the procedure above.
		\end{itemize} &
		\begin{itemize}
			\item There is only a \textit{key} sequence. No way to distinguish the "purpose" of each \textit{private key}.
		\end{itemize}\\
		\hline
	\end{tabular}
\end{center}
The \textit{type-2 deterministic wallet} it is an improvement of the \textit{type-1} because it has the same benefits (except for the need to back up two number instead of only one), but with a great advantage: it is possible to generate new addresses, obtained from the public keys, also in a non safe environment, having only $r$ and $MP$. 
\\ \\
A thief can only steal your privacy, if he steal only $MP$ and $r$. In fact he is able to see which messages (transaction) have you signed, without the possibility to sign new ones and spend your money.

\subsection{Deterministic Wallet \textit{type-3}}
Deterministic Wallet \textit{type-3} is the most elaborate among the ones considered. Starting from a \textit{seed} it is possible to obtain different \textit{keys} in a hierarchical way, with a structure similar to a tree. 
\\ \\
Let's see roughly how this wallet works:

\begin{enumerate}[label=(\roman*)]
	\item Generate a \textit{seed}, a random number from a \textit{Discrete Uniform Random Variable}, unique for each wallet.
	\begin{equation*}
	seed=X(\omega) \qquad X\sim \mathcal{U}(S)
	\end{equation*}
	Where $S$ is a finite set of natural number.
	\item Generate a \textit{master private key} from the \textit{seed}, using a stretching function: PBKDF2.
	\item From this \textit{master private key} it is possible to generate $2^{32}$ \textit{private key} using a irreversible hash function: SHA512
	\item All of this \textit{private key} "children" can derive $2^{32}$ \textit{private key} and all of these "grandchildren" can derive as many.
\end{enumerate}
This procedure can produce a huge number of keys. They seem independent from an outside point of view: it is impossible to guess that two keys are derived from the same \textit{seed}.
\\ \\ 
This particular type of Wallet is commonly known as \textbf{Hierarchical Deterministic Wallet}, one of the most used and widespread.
\\ \\
In the next chapter we will see in detail how it works.
