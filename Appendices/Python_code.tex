% Appendix B
\chapter{Python code}% Main appendix title
\label{AppendixB} % For referencing this appendix elsewhere, use 
This appendix shows the most relevant parts of the Python code made for this thesis. 

\begin{remark}
	These scripts only work if they are inserted in the repository of the professor Ferdinando M. Ametrano [\cite{13}].
\end{remark}

\section{Deterministic Wallet} 
First, let's show the Python code related to the first two type of Deterministic Wallets.

\subsection{Type-1}
\begin{flushleft}
	This is the script used to generate private and public keys, using the first type of deterministic derivation.
\end{flushleft}
\begin{lstlisting}[language=Python]
from secp256k1 import order, G, modInv, pointAdd, pointMultiply
from hashlib import sha256
import random

# secret random number
r = random.randint(0, order-1)
print('\nr =', hex(r), '\n')

# number of key pairs to generate
nKeys = 3
p = [0] * nKeys
P = [(0,0)] * nKeys

for i in range(0, nKeys):
	# H(i|r)
	H_i_r = int(sha256((hex(i)+hex(r)).encode()).hexdigest(), 16) %order
	p[i] = H_i_r
	P[i] = pointMultiply(p[i], G)
	print('prKey#', i, ':\n', hex(p[i]), sep='')
	print('PubKey#', i, ':\n', hex(P[i][0]), '\n',hex(P[i][1]),'\n', sep='')
\end{lstlisting}



\subsection{Type-2}
\begin{flushleft}
	This is the script used to generate private and public keys, using the second type of deterministic derivation.
\end{flushleft}
\begin{lstlisting}[language=Python]
from secp256k1 import order, G, modInv, pointAdd, pointMultiply
from hashlib import sha256
import random

# secret master private key
mp = random.randint(0, order-1)
print('\nsecret master private key:\n', hex(mp),'\n')

# public random number
r = random.randint(0, order-1)
print('public ephemeral key:\n', hex(r))

# Master PublicKey:
MP = pointMultiply(mp, G)
print('Master Public Key:\n', hex(MP[0]), '\n', hex(MP[1]), '\n')

# number of key pairs to generate
nKeys = 3
p = [0] * nKeys
P = [(0,0)] * nKeys

# PubKeys can be calculated without using privKeys
for i in range(0, nKeys):
	# H(i|r)
	H_i_r = int(sha256((hex(i)+hex(r)).encode()).hexdigest(), 16) %order
	P[i] = pointAdd(MP, pointMultiply(H_i_r, G))                 

# check that PubKeys match with privKeys
for i in range(0, nKeys):
	# H(i|r)
	H_i_r = int(sha256((hex(i)+hex(r)).encode()).hexdigest(), 16) %order
	p[i] = (mp + H_i_r) %order
	assert P[i] == pointMultiply(p[i], G)
	print('prKey#', i, ':\n', hex(p[i]), sep='')
	print('PubKey#', i, ':\n', hex(P[i][0]), '\n',hex(P[i][1]),'\n', sep='')
\end{lstlisting}


\section{Hierarchical Deterministic Wallet - BIP 32}

\begin{flushleft}
	This section shows the Python code related to the Hierarchical Deterministic Wallet, defined by BIP 32.
\end{flushleft}

\begin{lstlisting}[language=Python]
from secp256k1 import order, G, pointMultiply, pointAdd, a, b, prime
from hmac import HMAC
from hashlib import new as hnew
from hashlib import sha512, sha256
from base58 import b58encode_check, b58decode_check
from FiniteFields import modular_sqrt

BITCOIN_PRIVATE = b'\x04\x88\xAD\xE4'
BITCOIN_PUBLIC = b'\x04\x88\xB2\x1E'
TESTNET_PRIVATE = b'\x04\x35\x83\x94'
TESTNET_PUBLIC = b'\x04\x35\x87\xCF'
BITCOIN_SEGWIT_PRIVATE = b'\x04\xb2\x43\x0c'
BITCOIN_SEGWIT_PUBLIC = b'\x04\xb2\x47\x46'
PRIVATE = [BITCOIN_PRIVATE, TESTNET_PRIVATE, BITCOIN_SEGWIT_PRIVATE]
PUBLIC  = [BITCOIN_PUBLIC,  TESTNET_PUBLIC, BITCOIN_SEGWIT_PUBLIC]

def h160(inp):
	h1 = sha256(inp).digest()
	return hnew('ripemd160', h1).digest()

def bip32_isvalid_xkey(vbytes, depth, fingerprint, index, chain_code,key):
	assert len(key) == 33, "wrong length for key"
	if (vbytes in PUBLIC):
		assert key[0] in (2, 3)
	elif (vbytes in PRIVATE):
		assert key[0] == 0
	else:
		raise Exception("invalix key[0] prefix '%s'" % type(key[0]).__name__)
	assert int.from_bytes(key[1:33], 'big') < order, "invalid key"
	assert len(depth) == 1, "wrong length for depth"
	assert len(fingerprint) == 4, "wrong length for fingerprint"
	assert len(index) == 4, "wrong length for index"
	assert len(chain_code) == 32, "wrong length for chain_code"

def bip32_parse_xkey(xkey):
	decoded = b58decode_check(xkey)
	assert len(decoded) == 78, "wrong length for decoded xkey"
	info = {"vbytes": decoded[:4],
					"depth": decoded[4:5],
					"fingerprint" : decoded[5:9],
					"index" : decoded[9:13],
					"chain_code" : decoded[13:45],
					"key" : decoded[45:]
					}
	bip32_isvalid_xkey(info["vbytes"], info["depth"], info["fingerprint"], \
	info["index"], info["chain_code"], info["key"])
	return info

def bip32_compose_xkey(vbytes, depth, fingerprint, index, chain_code,key):
	bip32_isvalid_xkey(vbytes, depth, fingerprint, index, chain_code, key)
	xkey = vbytes + \
 				 depth + \
 				 fingerprint + \
 				 index + \
 				 chain_code + \
 				 key
	return b58encode_check(xkey)

def bip32_xprvtoxpub(xprv):
	decoded = b58decode_check(xprv)
	assert decoded[45] == 0, "not a private key"
	p = int.from_bytes(decoded[46:], 'big')
	P = pointMultiply(p, G)
	P_bytes = (b'\x02' if (P[1] % 2 == 0) else b'\x03') + P[0].to_bytes(32, 'big')
	network = PRIVATE.index(decoded[:4])
	xpub = PUBLIC[network] + decoded[4:45] + P_bytes
	return b58encode_check(xpub)

def bip32_master_key(seed, seed_bytes, vbytes = PRIVATE[0]):
	hashValue = HMAC(b"Bitcoin seed", seed.to_bytes(seed_bytes, 'big'), sha512).digest()
	p_bytes = hashValue[:32]
	p = int(p_bytes.hex(), 16) % order
	p_bytes = b'\x00' + p.to_bytes(32, 'big')
	chain_code = hashValue[32:]
	xprv = bip32_compose_xkey(vbytes, b'\x00', b'\x00\x00\x00\x00', b'\x00\x00\x00\x00', chain_code, p_bytes)
	return xprv

# Child Key Derivation
def bip32_ckd(extKey, child_index):
	parent = bip32_parse_xkey(extKey)
	depth = (int.from_bytes(parent["depth"], 'big') + 1).to_bytes(1, 'big')
	if parent["vbytes"] in PRIVATE:
		network = PRIVATE.index(parent["vbytes"])
		parent_prvkey = int.from_bytes(parent["key"][1:], 'big')
		P = pointMultiply(parent_prvkey, G)
		parent_pubkey = (b'\x02' if (P[1] % 2 == 0) else b'\x03') + P[0].to_bytes(32, 'big')
	else:
		network = PUBLIC.index(parent["vbytes"])
		parent_pubkey = parent["key"]
	fingerprint = h160(parent_pubkey)[:4]
	index = child_index.to_bytes(4, 'big')
	if (index[0] >= 0x80): #private (hardened) derivation
		assert parent["vbytes"] in PRIVATE, "Cannot do private (hardened) derivation from Pubkey"
		parent_key = parent["key"]
	else:
		parent_key = parent_pubkey
	hashValue = HMAC(parent["chain_code"], parent_key + index, sha512).digest()
	chain_code = hashValue[32:]
	p = int(hashValue[:32].hex(), 16)
	if parent["vbytes"] in PRIVATE:
		p = (p + parent_prvkey) % order
		p_bytes = b'\x00' + p.to_bytes(32, 'big')
		return bip32_compose_xkey(PRIVATE[network], depth, fingerprint, index, chain_code, p_bytes)
	else:
		P = pointMultiply(p, G)
		X = int.from_bytes(parent_pubkey[1:], 'big')
		Y_2 = X**3 + a*X + b
		Y = modular_sqrt(Y_2, prime)
		if (Y % 2 == 0):
			if (parent_pubkey[0] == 3):
				Y = prime - Y
		else:
			if (parent_pubkey[0] == 2):
				Y = prime - Y
		parentPoint = (X, Y)
		P = pointAdd(P, parentPoint)
		P_bytes = (b'\x02' if (P[1] % 2 == 0) else b'\x03') + P[0].to_bytes(32, 'big')
		return bip32_compose_xkey(PUBLIC[network], depth, fingerprint, index, chain_code, P_bytes)
\end{lstlisting}

\section{Mnemonic phrase}

In this last section there are shown the Python code related to the generation of the Mnemonic phrase from a given entropy and the way to obtain a seed from that set of words. 
\\ \\
Both the methods seen in this thesis have been implemented.


\subsection{BIP 39}
\begin{flushleft}
	These are the functions used to generate a valid BIP 39 Mnemonic phrase and the related seed.
\end{flushleft}

\begin{lstlisting}[language=Python]
from hashlib import sha256, sha512
from pbkdf2 import PBKDF2
import hmac

def from_entropy_to_mnemonic_int(entropy, ENT):  
	entropy_bytes = entropy.to_bytes(int(ENT/8), byteorder='big')
	checksum = sha256(entropy_bytes).digest()
	checksum_int = int.from_bytes(checksum, byteorder='big')
	checksum_bin = bin(checksum_int)
	while len(checksum_bin)<258:
		checksum_bin = '0b0' + checksum_bin[2:]
	entropy_bin = bin(entropy)
	while len(entropy_bin)<ENT+2:
		entropy_bin = '0b0' + entropy_bin[2:]
	entropy_checked = entropy_bin[2:] + checksum_bin[2:2+int(ENT/32)]
	number_mnemonic = (ENT/32 + ENT)/11
	assert number_mnemonic %1 == 0
	number_mnemonic = int(number_mnemonic)
	mnemonic_int = [0]*number_mnemonic
	for i in range(0,number_mnemonic):
		mnemonic_int[i] = int(entropy_checked[i*11:(i+1)*11],2)
	return mnemonic_int

def from_mnemonic_int_to_mnemonic(mnemonic_int, dictionary_txt):
	dictionary  = open(dictionary_txt, 'r').readlines()
	mnemonic = ''
	for j in mnemonic_int:
		mnemonic = mnemonic + ' ' +  dictionary[j][:-1]
	mnemonic = mnemonic[1:]
	return mnemonic
	
def generate_mnemonic_bip39(entropy, number_words = 24, dictionary = 'English_dictionary.txt'):
	ENT = int(number_words*32/3)
	mnemonic_int = from_entropy_to_mnemonic_int(entropy, ENT)
	mnemonic = from_mnemonic_int_to_mnemonic(mnemonic_int, dictionary)
	return mnemonic

def from_mnemonic_to_seed(mnemonic, passphrase = ''):
	PBKDF2_ROUNDS = 2048
	return PBKDF2(mnemonic, 'mnemonic' + passphrase, iterations = PBKDF2_ROUNDS, macmodule = hmac, digestmodule = sha512).read(64).hex())
\end{lstlisting}


\subsection{Electrum}

\begin{flushleft}
	These are the functions used to generate a valid Electrum Mnemonic phrase for a chosen version and the related seed.
\end{flushleft}

\begin{lstlisting}[language=Python]
from hashlib import sha512
from pbkdf2 import PBKDF2
import hmac
import binascii

def from_entropy_to_mnemonic_int_electrum(entropy, number_words):
	assert entropy < 2**(11*number_words)
	entropy_bin = bin(entropy)
	while len(entropy_bin)< number_words*11+2:
		entropy_bin = '0b0' + entropy_bin[2:]
	entropy_checked = entropy_bin[2:]
	mnemonic_int = [0]*number_words
	for i in range(0,number_words):
		mnemonic_int[i] = int(entropy_checked[i*11:(i+1)*11],2)
	return mnemonic_int

def from_mnemonic_int_to_mnemonic_electrum(mnemonic_int, dictionary_txt):
	dictionary  = open(dictionary_txt, 'r').readlines()
	mnemonic = ''
	for j in mnemonic_int:
		mnemonic = mnemonic + ' ' +  dictionary[j][:-1]
	mnemonic = mnemonic[1:]
	return mnemonic

def bh2u(x):
	return binascii.hexlify(x).decode('ascii')

def verify_mnemonic_electrum(mnemonic, version = "standard"):
	s = bh2u(hmac.new(b"Seed version", mnemonic.encode('utf8'), sha512).digest()) 
	if s[0:2] == '01':
		return version == "standard"
	elif s[0:3] == '100':
		return version == "segwit"
	elif s[0:3] == '101':
		return version == "2FA"
	else:
		return False
		
def generate_mnemonic_electrum(entropy, number_words = 24, version = "standard", dictionary = 'English_dictionary.txt'):
	is_verify = False
	while not is_verify:
		mnemonic_int = from_entropy_to_mnemonic_int_electrum(entropy, number_words)
		mnemonic = from_mnemonic_int_to_mnemonic_electrum(mnemonic_int, dictionary)
		is_verify = verify_mnemonic_electrum(mnemonic, version)
		if not is_verify:
			entropy = entropy + 1
	return mnemonic
	
def from_mnemonic_to_seed_eletrcum(mnemonic, passphrase=''):
	PBKDF2_ROUNDS = 2048
	return PBKDF2(mnemonic, 'electrum' + passphrase, iterations = PBKDF2_ROUNDS, macmodule = hmac, digestmodule = sha512).read(64).hex()
\end{lstlisting}

