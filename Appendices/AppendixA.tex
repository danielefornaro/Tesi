% Appendix A

\chapter{Bitcoin keys representation and addresses}% Main appendix title

\label{AppendixA} % For referencing this appendix elsewhere, use \ref{AppendixA}


In order to make it easy to store and recognise keys, some encods were designed.
\\ \\
A public key, a point in the EC, can be represented in two way: \textit{uncompressed} or \textit{compressed}.


\section{Uncompressed public key}
An uncompressed public key is rapresented in hexadecimal digits, and it is obtained simply concatenating the $x$ coordinate with the $y$ coordinate and adding $04$ at the beginning, for a total of $130$ hexadecimal digits. \\ \\
Example of an uncompressed public key: \\
0450863AD64A87AE8A2FE83C1AF1A8403CB53F53E486D8511DAD8A04887E5B235 22CD470243453A299FA9E77237716103ABC11A1DF38855ED6F2EE187E9C582BA6


\section{Compressed public key}
A compressed public key is obtained simply taking the $x$ coordinate and adding $02$ at the begging if the $y$ coordinate is even, $03$ otherwise. \\
This is due to the \textit{symmetry properties} of a point of the EC.
\\ \\
Example of a public key compressed:\\
0250863AD64A87AE8A2FE83C1AF1A8403CB53F53E486D8511DAD8A04887E5B2352


\section{WIF Private Key}
WIF stands for wallet import format and is the standard way used to write down a private key.
\begin{itemize}
	\item Add a version number ($80$ for Bitcoin) in front of the private key, in order to recognize quickly for what cryptocurrency that private key was used.
	\item Add $01$ at the end of the private key if you want a WIF \textit{compressed}, none if you want a WIF \textit{uncompressed}. The difference between these two types is that from a \textit{compressed} private key a \textit{compressed} public key is expected and from a \textit{uncompressed} private key a \textit{uncompressed} public key is expected.
	\item Addd a checksum at the end, obtained applying the SHA256 function twice to the string previously obtained, take the first 4 bytes (8 hexadecimal digits) and put them at the end of the string.
	\item Compute the Base58Encode, obtaining a 52 digit string.
\end{itemize}
Example of private key WIF: \\ KwdMAjGmerYanjeui5SHS7JkmpZvVipYvB2LJGU1ZxJwYvP98617

\section{Address}
Among the Bitcoin transactions, one of the most used is a \textit{Pay-to-PubkeyHash}, meaning that in the transaction you will not write directly the public key, but the hash of that public key.
\\
The hash function used in this freamwork is the HASH160 function, applied to the \textit{compressed} public key. This is an irreversible procedure, so you cannot obtain the public key from the public key hash. \\ \\
In order to obtain a valid Bitcoin address, it is needed to encode the \textit{PubkeyHash} in base58, adding first the version in front, the checksum at the end and then encode everything with Base58Encode, obtaining a 34 digit string. 
\\ \\
Example of an Address: \\
1BvBMSEYstWetqTFn5Au4m4GFg7xJaNVN2
