% Appendix A

\chapter{Bitcoin keys representation and addresses}% Main appendix title

\label{AppendixA} % For referencing this appendix elsewhere, use \ref{AppendixA}


In order to make it easy to store and recognize keys, in the Bitcoin framework, some encodes were designed.
\\ \\
In this appendix, we will briefly describe the possible ways to write down a public key, a private key and finally how it is possible to obtain a Bitcoin addresses (Pay-to-Public-Key-Hash).
\\ \\
All the examples below will start from the following private key (expressed in hexadecimal digits): \\
2AFEED53F26EF06521E7E825F83CB36A4632791A070A782E353230EAE71EBDD3.


\section{Public Key}
A public key is a point in the EC and can be represented in two way:
\begin{itemize}
	\item \textit{Uncompressed}.
	\item \textit{Compressed}.
\end{itemize}
 Both these encodes contains the same information and it is possible to obtain one from the other and \textit{vice versa}. 

\subsection{Uncompressed}
An uncompressed public key is a string of hexadecimal digits, obtained by concatenation of the $x$ coordinate with the $y$ coordinate (64 hexadecimal digits both). It is added $04$ at the beginning of the string, obtaining a total of $130$ hexadecimal digits. \\ \\
Example of an uncompressed public key: \\
049B7D40BA6BD08C0D1C46048279947AFEA89E6BA5E0C08AEEEBC3472F38C792B
72EC672B238AD98EECD29A2CD5F2465FEE3BB8205093CEBED8B94C8472FBA15E4.


\subsection{Compressed}
A compressed public key is a string of hexadecimal digits, it is obtained taking the $x$ coordinate and adding $02$ at the begging if the $y$ coordinate is even, $03$ otherwise. Its length will be of $66$ hexadecimal digits.

\begin{remark}
	The symmetry property of the EC allows us to write down only the x coordinate. The y coordinate can be derived by the equation of the EC that give us 2 possible y. The choice between these will be made base on the first two digits of the compressed public key.
\end{remark}
Example of a public key compressed:\\
029B7D40BA6BD08C0D1C46048279947AFEA89E6BA5E0C08AEEEBC3472F38C792B7.


\section{Private Key}
A private key is an integer number and in the Bitcoin framework, it is usually represented in a particular format: WIF (Wallet Import Format). 
\\ \\
In order to obtain a WIF Private Key the following procedure must be used:
\begin{itemize}
	\item Write down the private key in hexadecimal format. (64 digits).
	\item Add two digits as a version number ($80$ for Bitcoin) in front of the private key. This is done in order to recognize the purpose of the key.
	\item Add $01$ at the end of the private key if you want a WIF \textit{compressed}, none if you want a WIF \textit{uncompressed}. The difference between these two types is that from a \textit{compressed} private key a \textit{compressed} public key is expected to be derived and from a \textit{uncompressed} private key a \textit{uncompressed} public key is expected.
	\item Add a checksum at the end, obtained applying the SHA256 function twice to the string previously obtained, take the first 4 bytes (8 hexadecimal digits) and put them at the end of the string.
	\item Finally compute the encoding in base 58, obtaining a string of 51 digits if \textit{uncompressed} or 52 digits if \textit{compressed}.
\end{itemize}
A private key WIF compressed will start with the \textit{K} or \textit{L} and an uncompressed one will start with 5.
\\ \\
Example of private key WIF compressed: \\ KxfHiqW7h3N2puewVnHWBN4ucmBzK2iupiSUEGKjx1UNT8vvLwvP.
\\ \\
Example of the same private key shown above in WIF uncompressed: \\ 5J9DrMWFf5AJBQFwVGhSeEqTshNXCnhm96K1H9TT3VevM4iSudq.

\begin{remark}
	The two types of WIF private key give the same information. The only difference is in the public key that is expected to be derived from it. 
\end{remark}

\section{Address}
Among the Bitcoin transactions, one of the most used is a \textit{Pay-to-Public-Key-Hash}, meaning that in the transaction you will not write directly the public key, but the hash of that public key.
\\ \\
The hash function used in this framework is the HASH160, applied to the \textit{compressed} public key. The result is a \textit{PubkeyHash} and from the moment that the HASH160 is an irreversible function, it is infeasible to obtain the public key starting from the \textit{PubkeyHash}. 
\\ \\
In order to obtain a valid Bitcoin address, it is needed to encode the \textit{PubkeyHash} in base 58:

\begin{itemize}
	\item Write down the \textit{PubkeyHash} (160 bits).
	\item Add one byte as version ($00$ for Bitcoin) in front of the \textit{PubkeyHash}.
	\item Add a checksum at the end, obtained applying the SHA256 function twice to the string previously obtained, take the first 4 bytes and put them at the end of the string.
	\item Finally compute the encoding in base 58, obtaining a 34 digit string.
\end{itemize}
Example of an address: \\
1DFvgrsFE6qVfgX83E35SbLdpjiSFffY2q.

\begin{remark}
	This is not the only type of address in the Bitcoin framework, but it is the simplest one, derived directly from the public key. 
\end{remark}
