%%%%%%%%%%%%%%%%%%%%%%%%%%%%%%%%%%%%%%%%%
% Masters/Doctoral Thesis 
% LaTeX Template
% Version 2.5 (27/8/17)
%
% This template was downloaded from:
% http://www.LaTeXTemplates.com
%
% Version 2.x major modifications by:
% Vel (vel@latextemplates.com)
%
% This template is based on a template by:
% Steve Gunn (http://users.ecs.soton.ac.uk/srg/softwaretools/document/templates/)
% Sunil Patel (http://www.sunilpatel.co.uk/thesis-template/)
%
% Template license:
% CC BY-NC-SA 3.0 (http://creativecommons.org/licenses/by-nc-sa/3.0/)
%
%%%%%%%%%%%%%%%%%%%%%%%%%%%%%%%%%%%%%%%%%

%----------------------------------------------------------------------------------------
%	PACKAGES AND OTHER DOCUMENT CONFIGURATIONS
%----------------------------------------------------------------------------------------

\documentclass[
11pt, % The default document font size, options: 10pt, 11pt, 12pt
oneside, % Two side (alternating margins) for binding by default, uncomment to switch to one side
english, % ngerman for German
singlespacing, % Single line spacing, alternatives: onehalfspacing or doublespacing
%draft, % Uncomment to enable draft mode (no pictures, no links, overfull hboxes indicated)
%nolistspacing, % If the document is onehalfspacing or doublespacing, uncomment this to set spacing in lists to single
%liststotoc, % Uncomment to add the list of figures/tables/etc to the table of contents
toctotoc, % Uncomment to add the main table of contents to the table of contents
%parskip, % Uncomment to add space between paragraphs
%nohyperref, % Uncomment to not load the hyperref package
headsepline, % Uncomment to get a line under the header
%chapterinoneline, % Uncomment to place the chapter title next to the number on one line
%consistentlayout, % Uncomment to change the layout of the declaration, abstract and acknowledgements pages to match the default layout
]{MastersDoctoralThesis} % The class file specifying the document structure

\usepackage[utf8]{inputenc} % Required for inputting international characters
\usepackage[T1]{fontenc} % Output font encoding for international characters

\usepackage{mathpazo} % Use the Palatino font by default

\usepackage[backend=bibtex,style=authoryear,natbib=true]{biblatex} % Use the bibtex backend with the authoryear citation style (which resembles APA)

\addbibresource{example.bib} % The filename of the bibliography

\usepackage[autostyle=true]{csquotes} % Required to generate language-dependent quotes in the bibliography

\usepackage{amsmath}
\usepackage{amssymb}
\usepackage{hyperref}
\usepackage{enumitem}
\usepackage{listings}
\usepackage{color}
%\usepackage{natbib}

\usepackage{ntheorem}
\newtheorem*{remark}{Remark}
\newtheorem*{definition}{Definition}

\definecolor{codegreen}{rgb}{0,0.6,0}
\definecolor{codegray}{rgb}{0.5,0.5,0.5}
\definecolor{codepurple}{rgb}{0.58,0,0.82}
\definecolor{backcolour}{gray}{0.85}

\lstdefinestyle{mystyle}{
	backgroundcolor=\color{backcolour},   
	commentstyle=\color{codegreen},
	keywordstyle=\color{magenta},
	numberstyle=\tiny\color{codegray},
	stringstyle=\color{codepurple},
	basicstyle=\footnotesize,
	breakatwhitespace=false,         
	breaklines=true,                 
	captionpos=b,                    
	keepspaces=true,                 
	numbers=left,                    
	numbersep=5pt,                  
	showspaces=false,                
	showstringspaces=false,
	showtabs=false,                  
	tabsize=2
}

\lstset{style=mystyle}

%----------------------------------------------------------------------------------------
%	MARGIN SETTINGS
%----------------------------------------------------------------------------------------

\geometry{
	paper=a4paper, % Change to letterpaper for US letter
	inner=2.5cm, % Inner margin
	outer=3.8cm, % Outer margin
	bindingoffset=.5cm, % Binding offset
	top=1.5cm, % Top margin
	bottom=1.5cm, % Bottom margin
	%showframe, % Uncomment to show how the type block is set on the page
}

%----------------------------------------------------------------------------------------
%	THESIS INFORMATION
%----------------------------------------------------------------------------------------

\thesistitle{Elliptic Curve Hierarchical Deterministic Private Key Sequences: Bitcoin Standards and Best Practices} % Your thesis title, this is used in the title and abstract, print it elsewhere with \ttitle
\supervisor{Prof. Daniele \textsc{Marazzina}\\Prof. Ferdinando M. \textsc{Ametrano} } % Your supervisor's name, this is used in the title page, print it elsewhere with \supname
\examiner{} % Your examiner's name, this is not currently used anywhere in the template, print it elsewhere with \examname
\degree{Mathematical Engeneering} % Your degree name, this is used in the title page and abstract, print it elsewhere with \degreename
\author{Daniele \textsc{Fornaro}} % Your name, this is used in the title page and abstract, print it elsewhere with \authorname
\addresses{} % Your address, this is not currently used anywhere in the template, print it elsewhere with \addressname

\subject{Biological Sciences} % Your subject area, this is not currently used anywhere in the template, print it elsewhere with \subjectname
\keywords{} % Keywords for your thesis, this is not currently used anywhere in the template, print it elsewhere with \keywordnames
\university{\href{https://www.polimi.it}{Politecnico di Milano}} % Your university's name and URL, this is used in the title page and abstract, print it elsewhere with \univname
\department{\href{https://www.mate.polimi.it}{Department of Mathematics}} % Your department's name and URL, this is used in the title page and abstract, print it elsewhere with \deptname
%\group{\href{http://researchgroup.university.com}{Department of Mathematics}} % Your research group's name and URL, this is used in the title page, print it elsewhere with \groupname
\faculty{\href{http://www.ingindinf.polimi.it}{Industrial and Information Engineering}} % Your faculty's name and URL, this is used in the title page and abstract, print it elsewhere with \facname

\AtBeginDocument{
	\hypersetup{pdftitle=\ttitle} % Set the PDF's title to your title
	\hypersetup{pdfauthor=\authorname} % Set the PDF's author to your name
	\hypersetup{pdfkeywords=\keywordnames} % Set the PDF's keywords to your keywords
}

\begin{document}
	
	\frontmatter % Use roman page numbering style (i, ii, iii, iv...) for the pre-content pages
	
	\pagestyle{plain} % Default to the plain heading style until the thesis style is called for the body content
	
	%----------------------------------------------------------------------------------------
	%	TITLE PAGE
	%----------------------------------------------------------------------------------------
	
	\begin{titlepage}
		\begin{center}
			\includegraphics[width=3.5cm]{Logo_Politecnico_Milano} % University/department logo - uncomment to place it
			\\
			\vspace*{.03\textheight}
			{\scshape\LARGE \univname\par}\vspace{1.5cm} % University name
			\textsc{\Large Master Thesis}\\[0.5cm] % Thesis type
			
			\HRule \\[0.4cm] % Horizontal line
			{\huge \bfseries \ttitle\par}\vspace{0.4cm} % Thesis title
			\HRule \\[1.5cm] % Horizontal line
			
			\begin{minipage}[t]{0.35\textwidth}
				\begin{flushleft} \large
					\emph{Author:}\\
					\href{}{\authorname} % Author name - remove the \href bracket to remove the link
				\end{flushleft}
			\end{minipage}
			\begin{minipage}[t]{0.45\textwidth}
				\begin{flushright} \large
					\emph{Supervisors:} \\
					\href{}{\supname} % Supervisor name - remove the \href bracket to remove the link  
				\end{flushright}
			\end{minipage}\\[3cm]
			
			\vfill
			
			\large \textit{A thesis submitted in fulfillment of the requirements\\ for the degree of \degreename}\\[0.3cm] % University requirement text
			\textit{in the}\\[0.4cm]
			\facname\\\deptname\\[2cm] % Research group name and department name
			
			\vfill
			
			{\large 19 April 2018}\\[4cm] % Date
			
			%\includegraphics[width=2cm]{Logo_Politecnico_Milano} % University/department logo - uncomment to place it
			
			\vfill
		\end{center}
	\end{titlepage}
	
	%----------------------------------------------------------------------------------------
	%	DECLARATION PAGE
	%----------------------------------------------------------------------------------------
	
%	\begin{declaration}
%	\addchaptertocentry{\authorshipname} % Add the declaration to the table of contents
%	\noindent I, \authorname, declare that this thesis titled, \enquote{\ttitle} and the work presented in it are my own. I confirm that:
	
%	\begin{itemize} 
%		\item This work was done wholly or mainly while in candidature for a research degree at this University.
%		\item Where any part of this thesis has previously been submitted for a degree or any other qualification at this University or any other institution, this has been clearly stated.
%		\item Where I have consulted the published work of others, this is always clearly attributed.
%		\item Where I have quoted from the work of others, the source is always given. With the exception of such quotations, this thesis is entirely my own work.
%		\item I have acknowledged all main sources of help.
%		\item Where the thesis is based on work done by myself jointly with others, I have made clear exactly what was done by others and what I have contributed myself.\\
%	\end{itemize}
	
%	\noindent Signed:\\
%	\rule[0.5em]{25em}{0.5pt} % This prints a line for the signature
	
%	\noindent Date:\\
%	\rule[0.5em]{25em}{0.5pt} % This prints a line to write the date
%\end{declaration}
	
	\cleardoublepage
	
	%----------------------------------------------------------------------------------------
	%	QUOTATION PAGE
	%----------------------------------------------------------------------------------------
	
	\vspace*{0.2\textheight}
	
	\noindent\enquote{\itshape The Times 03/Jan/2009 Chancellor on brink of second bailout for banks}\bigbreak
	
	\hfill Bitcoin Blockchain
	
	%----------------------------------------------------------------------------------------
	%	ABSTRACT PAGE
	%----------------------------------------------------------------------------------------
	
	\begin{abstract}
		\addchaptertocentry{\abstractname} % Add the abstract to the table of contents
		The cryptography used by most of the cryptocurrencies is mainly based on the private-public key pair. It is therefore fundamental the method used to generate private keys, which must be efficient, secure and suitable for the situation. Among the various methods used by now, it has become a standard the one described in BIP32, the Hierarchical Deterministic Wallet. Starting from a sufficiently large random number, called SEED, it is possible to generate numerous private keys in a hierarchical and deterministic way through particular HASH functions and thanks to the elliptic curve properties. Several wallets also use a special algorithm to store the seed and to be able to back it up in a readable form, through the use of the mnemonic phrase, words selected from a specific dictionary. Although it is trying to reach consensus on a single standard to use, not all major players in the industry use the same method. This paper aims to clarify the various techniques used for the derivation of the keys, with particular attention to the HD wallet. It will also be analyzed the two principal way of encoding the seed, the one described into BIP39 as opposed to the proposal of Electrum, one of the main Bitcoin Wallet, highlighting their respective advantages and disadvantages.
		
	\end{abstract}
	
	%----------------------------------------------------------------------------------------
	%	ACKNOWLEDGEMENTS
	%----------------------------------------------------------------------------------------
	
	\begin{acknowledgements}
		\addchaptertocentry{\acknowledgementname} % Add the acknowledgements to the table of contents
		The acknowledgments and the people to thank go here, don't forget to include your project advisor\ldots
	\end{acknowledgements}
	
	%----------------------------------------------------------------------------------------
	%	LIST OF CONTENTS/FIGURES/TABLES PAGES
	%----------------------------------------------------------------------------------------
	\hypersetup{%
		colorlinks = true,
		linkcolor  = black
	}
	\tableofcontents % Prints the main table of contents
	
	%	\listoffigures % Prints the list of figures
	
	%	\listoftables % Prints the list of tables
	
	%----------------------------------------------------------------------------------------
	%	ABBREVIATIONS
	%----------------------------------------------------------------------------------------
	
	%	\begin{abbreviations}{ll} % Include a list of abbreviations (a table of two columns)
	
	%		\textbf{LAH} & \textbf{L}ist \textbf{A}bbreviations \textbf{H}ere\\
	%		\textbf{WSF} & \textbf{W}hat (it) \textbf{S}tands \textbf{F}or\\
	
	%	\end{abbreviations}
	
	%----------------------------------------------------------------------------------------
	%	PHYSICAL CONSTANTS/OTHER DEFINITIONS
	%----------------------------------------------------------------------------------------
	
%	\begin{constants}{lr@{${}={}$}l} % The list of physical constants is a three column table
		
		% The \SI{}{} command is provided by the siunitx package, see its documentation for instructions on how to use it
		
%		Speed of Light & $c_{0}$ & \SI{2.99792458e8}{\meter\per\second} (exact)\\
		%Constant Name & $Symbol$ & $Constant Value$ with units\\
		
%	\end{constants}
	
	%----------------------------------------------------------------------------------------
	%	SYMBOLS
	%----------------------------------------------------------------------------------------
	
%	\begin{symbols}{lll} % Include a list of Symbols (a three column table)
		
%		$a$ & distance & \si{\meter} \\
%		$P$ & power & \si{\watt} (\si{\joule\per\second}) \\
		%Symbol & Name & Unit \\
		
%		\addlinespace % Gap to separate the Roman symbols from the Greek
		
%		$\omega$ & angular frequency & \si{\radian} \\
		
%	\end{symbols}
	
	%----------------------------------------------------------------------------------------
	%	DEDICATION
	%----------------------------------------------------------------------------------------
	
	\dedicatory{For/Dedicated to/To my\ldots} 
	
	%----------------------------------------------------------------------------------------
	%	THESIS CONTENT - CHAPTERS
	%----------------------------------------------------------------------------------------
	
	\mainmatter % Begin numeric (1,2,3...) page numbering
	
	\pagestyle{thesis} % Return the page headers back to the "thesis" style
	
	% Include the chapters of the thesis as separate files from the Chapters folder
	% Uncomment the lines as you write the chapters
	
	%\include{Chapters/Chapter1}
	% Chapter 1

\chapter*{Introduction} % Main chapter title
\addcontentsline{toc}{chapter}{Introduction}
\label{Introduction} % For referencing the chapter elsewhere, use \ref{Chapter1} 

%----------------------------------------------------------------------------------------

% Define some commands to keep the formatting separated from the content 
%----------------------------------------------------------------------------------------
The cryptography used by most of the cryptocurrencies is mainly based on the
private-public key pair. It is therefore fundamental the method used to generate
private keys, which must be efficient, secure and suitable for the situation.
\\ \\
This thesis claim to analyze in detail the principal techniques used for the derivation of the public-private keys pair in the Bitcoin framework.
\\ \\
The first chapter it will give an explanation on the basic concepts that will be used for this work.
There are two fundamental elements used: HASH functions and the Elliptic Curve. The first are functions that are supposed to be irreversible, because it doesn't exist an analytic expression for the inverse. The only way to compute the inverse of a hash function is by trying and this will take to much time, due to high computational costs.
The second, the Elliptic Curve, is a plane algebraic curve defined by an equation, over a specific field. A point on this curve is called Public key, instead the integer number, used to obtain the point, is called private key. In this chapter all the most important properties of this curve will be explained.
\\ \\
In the second chapter will be analyzed in detail the principal techniques used in order to generate private and public key pairs. In particular we will see four type of derivations. The first and naive method consists of randomly extracting a number and consider it as a private key, then generate the corresponding public key each time a new pair is request. The other three methods that will be analyzed are the so called: \textit{deterministic}. This is due to the fact that in order to generate a bunch of keys, it is necessary one single datum, called \textit{seed}. These three methods are in a increasing scale of difficulty and complexity and we will see their principal advantage and disadvantage. The last type of derivation is the most used and derives the keys in a hierarchical way, this method will be seen in the next chapter.
\\ \\
The third chapter will be focus on the analysis of the Hierarchical Deterministic Wallets, the most sophisticated type of derivation used up to now. It is defined by BIP32 [\cite{1}] and it is used by most of the Bitcoin Wallet. This derivation is deterministic, a seed is needed, and it is hierarchical. From the seed it is possible to derive a large number of keys and all of these keys can derive new keys in the same way and so on. This procedure can be iterated as long as desired, leaving the user a wide choice in the derivation of these numbers.
\\ \\
In the fourth chapter there will be the analysis of two possible ways to store the seed: the first is the one proposed by BIP39 [\cite{2}] and it is the most used in the Bitcoin framework; the second is the one used by Electrum [\cite{3}], one of the principal Bitcoin wallet. Both of them used a mnemonic phrase, a sentence composed of a certain number of words from which it is possible to derive the seed. Nevertheless they have some differences and we will analyze them. The principal difference stands in the fact that both the methods have a way to verify the correctness of the mnemonic phrase. With BIP32 it is only possible to check if the phrase is plausible, instead with Electrum it is possible to assign a version to the seed that will be generated by the mnemonic phrase, giving a purpose to the keys generated from it.
\\ \\
The fifth chapter will be focused on some possible application of the Hierarchical Deterministic Wallet proposed by BIP32. In particular we will see the standard way to write a \textit{path}, in order to easily understand how to generate a particular key from the seed.  We will also analyze one of the standard used by most of the Bitcoin wallet: BIP43 [\cite{4}]. The purpose of this BIP is to give a particular meaning to some branches of the tree. We will therefore describe two important applications: multi-coin wallet BIP44 [\cite{5}] and SegWit addresses BIP49 [\cite{6}].
\\ \\
In the appendix there will be a summary of the methods used for the representation of the private and public keys in the Bitcoin framework and the respective addresses.
\\ \\
Along with this writing, I attach the github link of the repository of the python code for the course of the professor F. Ametrano. In this repository I have replicated in python all the procedures and methods presented and described in this thesis, neglecting all those parts that are not inherent to it and writing the important ones in a synthetic and essential way.
\\ \\
\hypersetup{
	colorlinks=true,
	urlcolor=black
}
\href{https://github.com/fametrano/BitcoinBlockchainTechnology}{\texttt{https://github.com/fametrano/BitcoinBlockchainTechnology}}

	% Chapter 1

\chapter{Elliptic Curve Geometry} % Main chapter title

\label{EC} % For referencing the chapter elsewhere, use \ref{Chapter1} 

%----------------------------------------------------------------------------------------

% Define some commands to keep the formatting separated from the content 
\newcommand{\keyword}[1]{\textbf{#1}}
\newcommand{\tabhead}[1]{\textbf{#1}}
\newcommand{\code}[1]{\texttt{#1}}
\newcommand{\file}[1]{\texttt{\bfseries#1}}
\newcommand{\option}[1]{\texttt{\itshape#1}}

%----------------------------------------------------------------------------------------

\section{Introduction}
Bitcoin security is based on public and private key cryptograpy. The main concept is that it is simple to compute the public key, knowing the private, but it is infeasible to calculate the private key, knowing the public. \\ \\
In order to obtain this result a particular Elliptic Curve is used.

\section{Elliptic Curve over $\mathbb{F}_p$}

A point $Q$, which coordinates are $x$ and $y$, belong to an Elliptic Curve if and only if $Q$ satisfies the following equation:
\begin{equation}\label{GeneralEC}
y^2=x^3+ax+b \quad \textrm{over} \ \mathbb{F}_p
\end{equation}
Where $\mathbb{F}_p$ is the finite field defined over the set of integers modulo $p$ and $a$ and $b$ are the coefficients of the curve. \\ \\
We can rewrite the equation \ref{GeneralEC} in the following way:

\begin{equation}\label{GeneralECmodp}
y^2=x^3+ax+b \quad \textrm{mod} \ p
\end{equation}
Figure \ref{fig:EC_ex} shows some examples of Elliptic Curve over $\mathbb{F}_p$ with $a=-7$ and $b=10$
\begin{figure}[ht!]
	\centering
	\includegraphics[width=9cm]{Figures/EC_ex.jpg}
	\caption{Points on the Elliptic Curve $y^2=x^3-7x+10 \; \textrm{mod} \ p$, with $p=19,97,127,487$ }
	\label{fig:EC_ex}
\end{figure}






\subsection{Operations}
A point on the Elliptic Curve has some particular properties:
\begin{itemize}
	\item Symmetry
	\item Point addition
	\item Scalar multiplication
\end{itemize}

\subsubsection{Symmetry}
For every point in the $x$ axis exists two points in the $y$ axis. Suppose that a point $P(x,y)$ belongs to the Elliptic Curve, then it must satisfy the equation \ref{GeneralEC}.
So it is easy to prove that the point $Q(x,p-y)$ belongs to the curve too. \\ \\
Furthermore we have $P=-Q$, from the moment that $P+Q=0$ (see addition below).

\subsubsection{Point addition}
We need to change our definition of addition in order to make it works in $\mathbb{F}_p$. 
In this framework we claim that if three points are aligned over the finite field $\mathbb{F}_p$, then they have zero sum. \\
So $P+Q=R$ if and only if $P$, $Q$ and $-R$ are aligned, in the sense\ shown in figure \ref{fig:EC_aligned}
\begin{figure}[ht!]
	\centering
	\includegraphics[width=9cm]{Figures/EC_aligned.jpg}
	\caption{Elliptic Curve $y^2=x^3-7x+10 \; \textrm{mod} \ 97$}
	\label{fig:EC_aligned}
\end{figure}


The equations for calculating point additions are the follow: \\
Suppose that \textit{A} and \textit{B} belong to the Elliptic Curve.

\begin{center}
	$ A=(x_1,y_1) \quad B=(x_2,y_2)$
\end{center}
Let's defined $ A+B :=(x_3,y_3) $ \\
So we have: 

\begin{center} 
	$s=\begin{cases} \dfrac{y_2-y_1}{x_2-x_1}, & \mbox{if } x_1\neq x_2 \\ \\ \dfrac{3x_1^2+a}{2y_1}, & \mbox{if } x_1= x_2\end{cases}$ 
\end{center}
\begin{center} 
	$ x_3=s^2-x_1-x_2  \quad$ mod $p$\\
	$y_3=s(x_1-x_3)-y_1  \quad$mod $p$
\end{center}

\subsubsection{Scalar multiplication}
Once defined the addition, any multiplication can be defined as:
\begin{center} 
	$ nP=\underbrace{
		P+P+\cdot \cdot \cdot+P
	}_{n\text{ times}}$
\end{center}
When $n$ is a very large number can be difficult or even infeasible to compute $nP$ in this way, but we can use the \textit{double and add algorithm} in order to perform multiplication in $\mathcal{O}(\log{}n)$ steps.

\subsection{Group order}
An elliptic curve defined over a finite field is a group and so it has a finite number of points. This number is called order of the group.\\
If the prime order is a very large number, it is impossible to count all the point in that field, but there is an algorithm that allows to calculate the order of a group in a fast and efficient way: \textit{Schoof's algorithm}.

\subsubsection{Cyclic subgroups}
Let's consider a generic point $P$, we have:
\begin{center} 
	$ nP+mP=\underbrace{
		P+\cdot \cdot \cdot+P
	}_{n\text{ times}}+
		\underbrace{
		P+\cdot \cdot \cdot+P
	}_{m\text{ times}}=
	\underbrace{
		P+\cdot \cdot \cdot+P
	}_{n+m\text{ times}} = 
	(n+m)P$
\end{center}
So multiple of $P$ are closed under addition and this is enough to prove that the set of the multiples of $P$ is a cyclic subgroup of the group formed by the elliptic curve.
\\ \\
The point $P$ is called generator of the cyclic subgroup.

\begin{remark}
	The order of $P$ is linked to the order of the elliptic curve by Lagrange's theorem, which states that the order of a subgroup is a divisor of the order of the parent group.
\end{remark}

\begin{remark}
	If the order of the group is a prime number, all the point $P$ generate a subgroup with the same order of the group.
\end{remark}

\subsection{Bitcoin Elliptic Curve}
Bitcoin uses a specific Elliptic Curve defined over the finite field of the natural numbers, where $a=0$ and $b=7$. \\ \\
The equation \ref{GeneralEC} becomes:

\begin{equation}\label{BitcoinEC}
y^2=x^3+7 \quad \textrm{mod} \ p
\end{equation}

The \textit{mod p} (modulo prime number) indicates that this curve is over a finite field of prime order $p=2^{256}-2^{32}-2^9-2^8-2^7-2^6-2^4-1$.

\section{Bitcoin private-public key cryptography}





	% Chapter 1

\chapter{Wallet} % Main chapter title

\label{bip32} % For referencing the chapter elsewhere, use \ref{Chapter1}

%----------------------------------------------------------------------------------------

% Define some commands to keep the formatting separated from the content 
%\newcommand{\keyword}[1]{\textbf{#1}}
%\newcommand{\tabhead}[1]{\textbf{#1}}
%\newcommand{\code}[1]{\texttt{#1}}
%\newcommand{\file}[1]{\texttt{\bfseries#1}}
%\newcommand{\option}[1]{\texttt{\itshape#1}}

%----------------------------------------------------------------------------------------


A Bitcoin wallet is a structure used to store keys. \\ \\
There are different type of wallet:
\begin{itemize}
	\item Nondeterministic (\textit{random}) Wallet
	\item Deterministic Wallet
\end{itemize}

\begin{remark}
	Bitcoin wallets contains keys, not coins. Coins are in the Blockchain.
\end{remark}

\section{Nondeterministic (\textit{random}) Wallet}
A nondeterministic wallet is the simplest type of wallet. Each Key is randomly and independently generated.

\begin{enumerate}[label=(\roman*)]
	\item Consider a \textit{Discrete Uniform Random Variable}
	\begin{equation*}
		X\sim \mathcal{U}(S)
	\end{equation*}
	Where $S$ is the finite set of natural number in the range from $1$ to $order$.
	\item Take some realizations $k_1,k_2...k_n$ of $X$ using enough entropy to make these numbesr (\textit{private keys}) impossible to guess.
	\begin{equation*}
		k_1=X(\omega_1) \quad  k_2=X(\omega_2) \quad ... \quad k_n=X(\omega_n)
	\end{equation*}
	\item Go back to point (i) every time new \textit{private keys} are needed.
\end{enumerate}

\subsection{Pros and Cons}
Let's focus on the good and bad aspect of this wallet.

\begin{center}
	\begin{tabular}{ |p{6cm}|p{6cm}|  }
	\hline
	\multicolumn{2}{|c|}{\textbf{\textit{Random Wallet}}} \\
	\hline \hline 
	\\
	\centerline{\textbf{Pros}}&\centerline{\textbf{Cons}}\\
	\hline
	\begin{itemize}
		\item Easy to implement
	\end{itemize} &
	\begin{itemize}
		\item Difficult to find \underline{real} new entropy for every new \textit{private key}.
		\item Every time new \textit{private keys} are needed, you need to make new back up.
		\item Difficult to store or back up in a \textit{non digital way}. Awkward to write it down all yours keys on a paper.
	\end{itemize}\\
	\hline
\end{tabular}
\end{center}

The use of \textit{random wallet} is strongly discouraged for anything other than simple test. There are no good reason to use it.

\section{Deterministic Wallets}
A deterministic wallet is a more sophisticated one, in which every key is generated from a common "\textit{seed}". This means that knowing the \textit{seed} means also to know all the keys in the wallet.\\ \\
There are different types of deterministic wallets, in this text we will analyze three main types:
\begin{itemize}
	\item Deterministic Wallet \textit{type 1}
	\item Deterministic Wallet \textit{type 2}
	\item Hierarchical Deterministic Wallet
\end{itemize}

These wallet are in increasing order of complexity.

\subsection{Deterministic Wallet \textit{type-1}}
The Deterministic Wallet \textit{type-1} is one of the simplest Wallet among the deterministic ones. Each key is generated adding a number in a sequential order to the \textit{seed} and then computing an \textit{hash} function such as the \textbf{SHA256} function.
\\ \\
Let's see how it works:

\begin{enumerate}[label=(\roman*)]
	\item Generate a \textit{seed}, a random number from a \textit{Discrete Uniform Random Variable}
	\begin{equation*}
	seed=X(\omega) \qquad X\sim \mathcal{U}(S)
	\end{equation*}
	Where $S$ is the finite set of natural number in the range from $1$ to $order$.
	\item
\end{enumerate}


\subsection{Pros and Cons}

\section{Hierarchical deterministic Wallets}

\subsection{Key Concept}

\subsubsection{Seed}

\subsubsection{Extended Key}

\subsubsection{Mnemonic and Passphrase}

\subsection{Pros and Cons}


	% Chapter 1

\chapter{Hierarchical Deterministic Wallet} % Main chapter title

\label{hd wallet} % For referencing the chapter elsewhere, use \ref{Chapter1}

%----------------------------------------------------------------------------------------

% Define some commands to keep the formatting separated from the content 
%\newcommand{\keyword}[1]{\textbf{#1}}
%\newcommand{\tabhead}[1]{\textbf{#1}}
%\newcommand{\code}[1]{\texttt{#1}}
%\newcommand{\file}[1]{\texttt{\bfseries#1}}
%\newcommand{\option}[1]{\texttt{\itshape#1}}

%----------------------------------------------------------------------------------------

In this chapter we will see how an HD wallet works.
\\ \\
\section{Elements}
Let's focus on the main elements of the Wallet:
\begin{itemize}[label=$\diamond$]
	\item Seed
	\item Extended keys
\end{itemize}

\subsection{Seed}
The entire Wallet is based on a \textit{seed}.
\\ \\
It is a number taken from a \textit{Discrete Uniform Random Variable}
\begin{equation*}
seed=X(\omega) \qquad X\sim \mathcal{U}(S)
\end{equation*}
Where $S$ is the finite set of natural number in the range from $1$ to an arbitrary value.\\ Obviously the greater the set from which the number can be extracted, the better it is for the security of the seed itself.
\\ \\
This is an example of seed expressed in hexadecimal format: \\
\textit{seed}=fffcf9f6f3f0edeae7e4e1dedbd8d5d2cfccc9c6c3c0bdbab7b4b1aeaba8a5a29f9c999 \\ 693908d8a8784817e7b7875726f6c696663605d5a5754514e4b484542 

\subsection{Extended Key}
An Extended Key is a sequence of bytes, encoded in base58. It contains all the information necessary for the derivation. When the derivation is made for the first time from the seed, the extended key is called master key.\\ \\
Once it is decoded we will obtain exactly 78 bytes, with a specific meaning and order:
\begin{itemize}[label=$\circledast$]
	\item 4 bytes are used to specified the \textbf{version}.
	\item 1 byte is used to specified the \textbf{depth} in the hierarchical tree: the extended key derived directly from the seed has $depth=0$, its first children have $depth=1$, grandchildren have $depth=2$ and so on.
	\item 4 bytes are used for the \textbf{fingerprint}. It is a unique value that identify the parent. Compute the HASH160 function on the "parent" public key in a compressed form and then take the first 4 bytes, this is the fingerprint of the child:
	\begin{equation*}
	fingerprint=HASH160(\text{parent public key})[0:4]
	\end{equation*}
	Where $[0:4]$ is a python notation.\\
	For the master key the fingerprint is formed by 4 zeros bytes: $fingerprint=0000000000$
	\item 4 bytes are used to specified the \textbf{index} of the child. \\
	For the master key the index is formed by 4 zeros bytes: $index=0000000000$
	\item 32 bytes are used for the \textbf{chain code}. The chain code is used in order to introduce a sort of entropy in the children generation. We will see below how it works.
	\item 33 bytes are used for the \textbf{key}. It can be \textit{private} or \textit{public}. \\ Public key is expressed in compact form, so the first byte is always $02$ or $03$. The first byte of the private key is always $00$ in order to distinguish the key from the public one.\\
\end{itemize}
An extended key is called \textbf{Extended Private Key} if the lasts 33 bytes are used to specify the private key; it is called \textbf{Extended Public Key} if they are used to specify the public key.
\\ \\
For the Bitcoin mainnet it is used for the \textbf{version}: $0x0488ADE4$ for an extended private key, $0x0488B21E$ for an extended public key. When this bytes are encoded in base58, they returns \textit{xprv} and \textit{xpub} respectively.
 
\section{From SEED to Master Private Key}
In this section we will see in detail how it is possible to switch from a \textit{seed} to a \textit{master private key}. \\ \\
First of all we need to convert the seed into a string of bytes, where the most significant bytes come first (big endian). In order to do so, we need to know how much long we want the string of bytes. \\ \\
Let's see a practical example:
\begin{equation*}
\begin{split}
&byte\_string_1=00\; 00\; 00 \; 01 \\
&byte\_string_2=00\; 00\; 01 \\
&byte\_string_3=00\; 01 \\
&byte\_string_4=01
\end{split}
\end{equation*}
These $4$ byte strings are obtained from the same seed: $seed=1$ and the only different is the length of the string.
\begin{remark}
	Different length of string produce different master private key, even if the seed is the same number.
\end{remark}
In python:
\begin{lstlisting}[language=Python]
byte_string = seed.to_bytes(seed_bytes, 'big')
\end{lstlisting}
Where $seed$ is a \textit{int}, \textit{seed\_bytes} is the number of bytes that the \textit{byte\_string} should have. \\ \\
It is essential to specify the length of the byte string, otherwise there will be obtained different wallets. \\ \\
Once we obtain a string of bytes, we will compute the HMAC algorithm. The hash function used for HMAC is the SHA512 and the \textit{key} is a particular string of bytes: \textit{b"Bitcoin seed"}. In python the implementation is the follow:

\begin{lstlisting}[language=Python]
from hashlib import sha512
from hmac import HMAC

hashValue = HMAC(b"Bitcoin seed", byte_string, sha512).digest()
\end{lstlisting}
Where \textit{.digest()} is used in order to return a string of bytes.
\\ \\
Now we have obtained an \textit{hashValue} of $512$ bits, so $64$ bytes. Consider the firsts $32$ bytes as the master private key and the next $32$ bytes as the master chain code. A python implementation is the follow:

\begin{lstlisting}[language=Python]
private_key_bytes = hashValue[0:32]
chain_code_bytes = hashValue[32:64]
\end{lstlisting}
\begin{flushleft}
	Now we have two byte strings, one for the master private key and the other for the master chain code.
\end{flushleft}
It is important to remember that a private key must be in the range between $1$ and $order$, so the byte string for the private key should be converted in \textit{int} and then take the \textit{mod order}. In python we have:

\begin{lstlisting}[language=Python]
private_key = int(private_key_bytes.hex(), 16) % order
\end{lstlisting}
\begin{flushleft}
	Finally we will concatenate all the informations obtained in order to form a Master Extended Private Key (in bytes format):
\end{flushleft}

\begin{itemize}
	\item vbytes = $b'\backslash x04\backslash x88\backslash xAD\backslash xE4'$
	\item depth = $b'\backslash x00'$
	\item fingerprint = $b'\backslash x00\backslash x00\backslash x00\backslash x00'$
	\item index = $b'\backslash x00\backslash x00\backslash x00\backslash x00'$
	\item chain code is the one previously computed
	\item private key = $b'\backslash x00'$ $+$ private key in bytes format, previously computed.
\end{itemize}
Then the Master Extended Private Key is formed by concatenation:

\begin{lstlisting}[language=Python]
xkey = vbytes + depth + fingerprint + index + chain_code + key
\end{lstlisting}
\begin{flushleft}
	In order to make it readable, a base58 encode is performed.
\end{flushleft}
This is an example of Master Extended Private Key: \\
xprv9s21ZrQH143K3wEaiSJZ8jYCuZF1oJoXHiwFcx2WwXqQHD4ZLdyEAFZ22M4 BmQT82HRbWssLArj53YDQTj6vSN4iH6nTiSQ61C5CckxUtDq

\begin{remark}
	The SHA512 is an irreversible function, so it is infeasible to obtained the seed, knowing the Extended Key. (It is also useless because with the master key you can derive all the keys in the wallet).
\end{remark}

\section{Child Key derivation}
From any extended private key it is possible to obtain different child keys. There are two method used in order to do so:
\begin{itemize}
	\item Normal
	\item Hardened
\end{itemize}
Both methods have some advantage and disadvantage that we will discuss later. For every situation it is essential to use the method that best fit.
\\ \\
For both the method the derivation starts from a Extended Private Key. From this key some essential information are necessary:
\begin{itemize}[label=$\star$]
	\item Chain code
	\item Private key
\end{itemize}
It is also required a number, used in order to specify the \textbf{index} of the child. This number should be in the range between $0$ and $4294967295$. This is due to the fact that in any Extended Key there are 4 bytes used to specify the index of the child:
\begin{equation*}
	max \; index=(FF\;FF\;FF\;FF)_{16} = (4294967295)_{10}
\end{equation*}
In fact it is possible to generate even a greater number of children from the same parent, but it would not be possible to write the corresponding Extended Key in the format described above.

\section{Normal derivation}

As already mentioned we need only 3 ingredient in order to derive a key. Let's see how we can combine them together in order to obtain a new key (and chain code). \\ \\
First we need to compute the Parent Public Key $P$. This is obtained from the usual scalar multiplication of an EC point (the Generator $G$) with the Parent Private Key $p$:

\begin{equation*}
P=p\cdot G
\end{equation*}
Consider only the compress form of $P$ and convert this value into a byte string, obtaining 33 bytes.
\\ \\
Concatenate this number 33 byte string to the 4 byte string representing the index number:

\begin{equation*}
msg = compressed \; public\;key \;|\; index
\end{equation*}
$msg$ is a string of $37$ bytes. \\ \\
Apply the HMAC algorithm with the following input:

\begin{itemize}[label=$\odot$]
	\item \textbf{Hash function}: SHA512
	\item \textbf{Key}: chain code
	\item \textbf{Message}: $msg$
\end{itemize}
The Python code is the follow:
\begin{lstlisting}[language=Python]
from hmac import HMAC
from hashlib import sha512

msg=parent_key + index
hashValue = HMAC(parent_chain_code, msg, sha512).digest()
\end{lstlisting}
\begin{flushleft}
	The result is a string of 64 bytes: \textit{hashValue}.
\end{flushleft}
Split this string of bytes in two: the last 32 are the child chain code. Take the first 32 bytes, convert them into an integer number and sum it to the parent private key (mod \textit{order}), obtaining the child private key.\\ \\
This is the python code:

\begin{lstlisting}[language=Python]
child_chain_code = hashValue[32:]
p = int(hashValue[:32].hex(), 16)
child_private_key = (p + parent_private_key) % order
\end{lstlisting}
	% Chapter 1

\chapter{Mnemonic phrase} % Main chapter title

\label{mnemonic} % For referencing the chapter elsewhere, use \ref{Chapter1}

%----------------------------------------------------------------------------------------

% Define some commands to keep the formatting separated from the content 
%\newcommand{\keyword}[1]{\textbf{#1}}
%\newcommand{\tabhead}[1]{\textbf{#1}}
%\newcommand{\code}[1]{\texttt{#1}}
%\newcommand{\file}[1]{\texttt{\bfseries#1}}
%\newcommand{\option}[1]{\texttt{\itshape#1}}

%----------------------------------------------------------------------------------------

We have seen how it is possible to generate keys starting from a seed. But a seed is a long number, difficult to remember and not easy to write down on a paper. You may incur typos while transcribing it, and this can compromise the entire wallet.
\begin{remark}
	Mistyping a single digit in the seed produce completely different keys
\end{remark}
In order to work around this problem, some solution were implemented. Among them, the most widespread and used is the one described by BIP39, \textit{Bitcoin Improvement Proposal}. This is not the only one, in this chapter we will also see another solution proposed by Electrum, one of the most famous Bitcoin wallet.
\\ \\
Both these solutions used a \textit{Mnemonic phrase}, from which the seed is obtained. This sentence is designed to avoid typing errors while maintaining the same level of security and entropy.
\begin{flushleft}
	\textbf{What is a Mnemonic phrase?}
\end{flushleft}
A Mnemonic phrase is a set of words taken from a specific dictionary. Although the choice of the dictionary is not binding, the most commonly used among the practitioners is the English one, defined by BIP39. It contains 2048 common words of the English language, each of them from 3 to 9 letters. The set of words that make up the dictionary must be chosen in such a way that the words within it are easy to remember and difficult to misinterpret with one another. It is better to avoid to insert into the dictionary two words with similar meaning or spelling.

\section{BIP39}
First we will see what is and how to generate a mnemonic phrase in the framework of BIP39 and then how it is possible to obtain a seed from it.

\subsection{Mnemonic Generation}
In order to generate a Mnemonic phrase we will start from a given entropy, that can be seen as a large integer number. The way to obtain it can be left free to the user: he can obtain it by inserting arbitrarily chosen numbers (poor choice of randomness), roll a die many times or with any other method he considers suitable. Many softwares provide a function that generate entropy with quite randomness, but if someone is skeptical about the reliability of software randomness, he must provide himself with such integer number.
\\ \\
Call \textit{ENT} the number of binary digits of the given entropy. Then \textit{ENT} should belongs to a given set:
\begin{equation*}
	ENT \in \{128,160,192,224,256\} =: entropy \; digits 
\end{equation*}
The reason for a given length for the entropy will be clear in a moment.
\\ \\
Write the entropy in bytes format, obtaining a string of $ENT/8$ length. Compute the SHA256 algorithm and consider only the first $ENT/32$ bits as a checksum. Finally add these bits to the bottom of the entropy, obtaining an integer number, called \textit{entropy\_checked}, expressed in binary format of length equal to: $ENT+ENT/32$.

\begin{flushleft}
	In python:
\end{flushleft}

\begin{lstlisting}[language=Python]
from hashlib import sha256

entropy_bytes = entropy.to_bytes(int(ENT/8), byteorder='big')
checksum = sha256(entropy_bytes).digest()
entropy_checked = entropy_bin + checksum_bin[:int(ENT/32)]
\end{lstlisting}
Where \textit{entropy\_bin} and \textit{checksum\_bin} are strings of bits that can be concatenated.
\\ \\
Now it is clear the reason for a constraint on the length of the entropy in input: 
\begin{enumerate}[label=\roman*]
	\item \textit{ENT} must be a dividend of 32
	\item $ENT<128$ could be not secure enough.
	\item $ENT>256$ is useless.
\end{enumerate}
The point (i) is due to the structure proposed by BIP39. It is only a convention to take the first $ENT/32$ bits as a checksum. However it is essential that the final length of the entropy plus the checksum must be a dividend of 11, from the moment that the dictionary is a set of $2^{11}$ words.
\\ \\
The point (ii) is just a suggestion because take less entropy could bring to a leak of security, from the moment that it will be easier for an attacker to guess your mnemonic phrase by trying out all the possible combinations. It is important to remember that adding even a single bit of entropy, doubles the difficulty of guessing it.








	% Chapter 1

\chapter{How to use a HD Wallet} % Main chapter title

\label{bip32} % For referencing the chapter elsewhere, use \ref{Chapter1}

%----------------------------------------------------------------------------------------

% Define some commands to keep the formatting separated from the content 
%\newcommand{\keyword}[1]{\textbf{#1}}
%\newcommand{\tabhead}[1]{\textbf{#1}}
%\newcommand{\code}[1]{\texttt{#1}}
%\newcommand{\file}[1]{\texttt{\bfseries#1}}
%\newcommand{\option}[1]{\texttt{\itshape#1}}

%----------------------------------------------------------------------------------------

\section{BIP 44}





	% Chapter 1

\chapter*{Conclusion} % Main chapter title
\addcontentsline{toc}{chapter}{Conclusion}
\label{Conclusion} % For referencing the chapter elsewhere, use \ref{Chapter1} 

%----------------------------------------------------------------------------------------

% Define some commands to keep the formatting separated from the content 
%----------------------------------------------------------------------------------------
The main purpose of this work has been the analysis of methods used to generate a sequence of private and public keys and to store the seed from which the sequence is derived. 
\\ \\
First, we have briefly described some simple deterministic derivation, then we have analyzed the Hierarchical Deterministic Wallet. It is possible to derive an extended key in two way: normal and hardened. The use of the normal derivation allows public-to-public derivation, that is the derivation of a sequence of public keys from an extended public key, without access to any private key; anyway the entire wallet is compromised if both a parent extended public key and a child extended private key are stolen. The use of the hardened derivation prevented this problem, but it does not allow public-to-public derivation. 
\\ \\
Then we focused on the two methods mostly used to generate the seed: the version proposed by BIP39 and the one proposed by Electrum. Both of them start from a given entropy to generate a mnemonic phrase, which is then used to obtain a seed. The two methods are very similar, but with some subtle differences. In this thesis these differences have been analyzed, showing pros and cons of each method. 
\\ \\
It was not a goal of this work to point out a better proposal, but to provide a complete and detailed overview of the various way to generate asymmetric cryptographic keys.
\vfill
\textit{"We often fear what we do not understand. Our best defense is knowledge."}
\begin{flushright}
	Lieutenant Tuvok, Star Trek: Voyager
\end{flushright}


	
	%----------------------------------------------------------------------------------------
	%	THESIS CONTENT - APPENDICES
	%----------------------------------------------------------------------------------------
	
	\appendix % Cue to tell LaTeX that the following "chapters" are Appendices
	
	% Include the appendices of the thesis as separate files from the Appendices folder
	% Uncomment the lines as you write the Appendices
	
	% Appendix A

\chapter{Bitcoin keys representation and addresses}% Main appendix title

\label{AppendixA} % For referencing this appendix elsewhere, use \ref{AppendixA}


In order to make it easy to store and recognise keys, some encods were designed.
\\ \\
A public key, a point in the EC, can be represented in two way: \textit{uncompressed} or \textit{compressed}.


\section{Uncompressed public key}
An uncompressed public key is rapresented in hexadecimal digits, and it is obtained simply concatenating the $x$ coordinate with the $y$ coordinate and adding $04$ at the beginning, for a total of $130$ hexadecimal digits. \\ \\
Example of an uncompressed public key: \\
0450863AD64A87AE8A2FE83C1AF1A8403CB53F53E486D8511DAD8A04887E5B235 22CD470243453A299FA9E77237716103ABC11A1DF38855ED6F2EE187E9C582BA6


\section{Compressed public key}
A compressed public key is obtained simply taking the $x$ coordinate and adding $02$ at the begging if the $y$ coordinate is even, $03$ otherwise. \\
This is due to the \textit{symmetry properties} of a point of the EC.
\\ \\
Example of a public key compressed:\\
0250863AD64A87AE8A2FE83C1AF1A8403CB53F53E486D8511DAD8A04887E5B2352


\section{WIF Private Key}
WIF stands for wallet import format and is the standard way used to write down a private key.
\begin{itemize}
	\item Add a version number ($80$ for Bitcoin) in front of the private key, in order to recognize quickly for what cryptocurrency that private key was used.
	\item Add $01$ at the end of the private key if you want a WIF \textit{compressed}, none if you want a WIF \textit{uncompressed}. The difference between these two types is that from a \textit{compressed} private key a \textit{compressed} public key is expected and from a \textit{uncompressed} private key a \textit{uncompressed} public key is expected.
	\item Addd a checksum at the end, obtained applying the SHA256 function twice to the string previously obtained, take the first 4 bytes (8 hexadecimal digits) and put them at the end of the string.
	\item Compute the Base58Encode, obtaining a 52 digit string.
\end{itemize}
Example of private key WIF: \\ KwdMAjGmerYanjeui5SHS7JkmpZvVipYvB2LJGU1ZxJwYvP98617

\section{Address}
Among the Bitcoin transactions, one of the most used is a \textit{Pay-to-PubkeyHash}, meaning that in the transaction you will not write directly the public key, but the hash of that public key.
\\
The hash function used in this freamwork is the HASH160 function, applied to the \textit{compressed} public key. This is an irreversible procedure, so you cannot obtain the public key from the public key hash. \\ \\
In order to obtain a valid Bitcoin address, it is needed to encode the \textit{PubkeyHash} in base58, adding first the version in front, the checksum at the end and then encode everything with Base58Encode, obtaining a 34 digit string. 
\\ \\
Example of an Address: \\
1BvBMSEYstWetqTFn5Au4m4GFg7xJaNVN2

	%\include{Appendices/AppendixB}
	%\include{Appendices/AppendixC}
	
	%----------------------------------------------------------------------------------------
	%	BIBLIOGRAPHY
	%----------------------------------------------------------------------------------------
	
	\begin{thebibliography}{12}
		\bibitem{1}
		\href{https://github.com/bitcoin/bips/blob/master/bip-0032.mediawiki}{BIP32, Bitcoin Improvement Proposal number 32 \\ \texttt{https://github.com/bitcoin/bips/blob/master/bip-0032.mediawiki}}
		
		\bibitem{2}
		\href{https://github.com/bitcoin/bips/blob/master/bip-0039.mediawiki}{BIP39, Bitcoin Improvement Proposal number 39 \\ \texttt{https://github.com/bitcoin/bips/blob/master/bip-0039.mediawiki}}
		
		
		\bibitem{3} 
		\href{https://electrum.org}{Electrum Bitcoin Wallet \\ \texttt{https://electrum.org}}
		
		\bibitem{4} 
		\href{https://github.com/bitcoin/bips/blob/master/bip-0043.mediawiki}{BIP43, Bitcoin Improvement Proposal number 43 \\ \texttt{https://github.com/bitcoin/bips/blob/master/bip-0043.mediawiki}}

		\bibitem{5}
		\href{https://github.com/bitcoin/bips/blob/master/bip-0044.mediawiki}{BIP44, Bitcoin Improvement Proposal number 44 \\ \texttt{https://github.com/bitcoin/bips/blob/master/bip-0044.mediawiki}}
		
		\bibitem{6} 
		\href{https://github.com/bitcoin/bips/blob/master/bip-0049.mediawiki}{BIP49, Bitcoin Improvement Proposal number 49 \\ \texttt{https://github.com/bitcoin/bips/blob/master/bip-0049.mediawiki}}
		
		\bibitem{7} 
		\href{https://github.com/bitcoinbook/bitcoinbook}{Andreas M. Antonopoulos, \textit{Mastering Bitcoin 2nd Edition - Programming the Open Blockchain}. 2017.}
		
		\bibitem{8} 
		\href{http://andrea.corbellini.name}{Andre Corbellini, \texttt{http://andrea.corbellini.name}}
		
		\bibitem{9} 
		Bundesamt fur Sicherheit in der Informationstechnik, \textit{Elliptic Curve Cryptography}, 2007.
		
		\bibitem{10} 
		Christof Paar, Jan Pelzl, \textit{Understanding Cryptography}, 2010.
		
		\bibitem{11} 
		Satoshi Nakamoto, \textit{Bitcoin: A Peer-to-Peer Electronic Cash System}, 2008.
		
		\bibitem{12} 
		\href{https://bitcoincore.org}{Bitcoin Core, \texttt{https://bitcoincore.org}}
		

	\end{thebibliography}

%\bibliographystyle{unsrt}
%\bibliography{bibliography}
	
	\printbibliography[heading=bibintoc]
	
	%----------------------------------------------------------------------------------------
	
\end{document}  